% \documentclass{article}
% \usepackage[utf8]{inputenc}

% \title{MC102-2018s1-Aula09-takeaway}
% \author{Arthur Catto}
% \date{March 2018}


% Default to the notebook output style

    


% Inherit from the specified cell style.




    

% Default to the notebook output style

    


% Inherit from the specified cell style.




    
\documentclass[11pt]{article}

    
    
    \usepackage[T1]{fontenc}
    % Nicer default font (+ math font) than Computer Modern for most use cases
    \usepackage{mathpazo}

    % Basic figure setup, for now with no caption control since it's done
    % automatically by Pandoc (which extracts ![](path) syntax from Markdown).
    \usepackage{graphicx}
    % We will generate all images so they have a width \maxwidth. This means
    % that they will get their normal width if they fit onto the page, but
    % are scaled down if they would overflow the margins.
    \makeatletter
    \def\maxwidth{\ifdim\Gin@nat@width>\linewidth\linewidth
    \else\Gin@nat@width\fi}
    \makeatother
    \let\Oldincludegraphics\includegraphics
    % Set max figure width to be 80% of text width, for now hardcoded.
    \renewcommand{\includegraphics}[1]{\Oldincludegraphics[width=.8\maxwidth]{#1}}
    % Ensure that by default, figures have no caption (until we provide a
    % proper Figure object with a Caption API and a way to capture that
    % in the conversion process - todo).
    \usepackage{caption}
    \DeclareCaptionLabelFormat{nolabel}{}
    \captionsetup{labelformat=nolabel}

    \usepackage{adjustbox} % Used to constrain images to a maximum size 
    \usepackage{xcolor} % Allow colors to be defined
    \usepackage{enumerate} % Needed for markdown enumerations to work
    \usepackage{geometry} % Used to adjust the document margins
    \usepackage{amsmath} % Equations
    \usepackage{amssymb} % Equations
    \usepackage{textcomp} % defines textquotesingle
    % Hack from http://tex.stackexchange.com/a/47451/13684:
    \AtBeginDocument{%
        \def\PYZsq{\textquotesingle}% Upright quotes in Pygmentized code
    }
    \usepackage{upquote} % Upright quotes for verbatim code
    \usepackage{eurosym} % defines \euro
    \usepackage[mathletters]{ucs} % Extended unicode (utf-8) support
    \usepackage[utf8x]{inputenc} % Allow utf-8 characters in the tex document
    \usepackage{fancyvrb} % verbatim replacement that allows latex
    \usepackage{grffile} % extends the file name processing of package graphics 
                         % to support a larger range 
    % The hyperref package gives us a pdf with properly built
    % internal navigation ('pdf bookmarks' for the table of contents,
    % internal cross-reference links, web links for URLs, etc.)
    \usepackage{hyperref}
    \usepackage{longtable} % longtable support required by pandoc >1.10
    \usepackage{booktabs}  % table support for pandoc > 1.12.2
    \usepackage[inline]{enumitem} % IRkernel/repr support (it uses the enumerate* environment)
    \usepackage[normalem]{ulem} % ulem is needed to support strikethroughs (\sout)
                                % normalem makes italics be italics, not underlines
    

    
    
    % Colors for the hyperref package
    \definecolor{urlcolor}{rgb}{0,.145,.698}
    \definecolor{linkcolor}{rgb}{.71,0.21,0.01}
    \definecolor{citecolor}{rgb}{.12,.54,.11}

    % ANSI colors
    \definecolor{ansi-black}{HTML}{3E424D}
    \definecolor{ansi-black-intense}{HTML}{282C36}
    \definecolor{ansi-red}{HTML}{E75C58}
    \definecolor{ansi-red-intense}{HTML}{B22B31}
    \definecolor{ansi-green}{HTML}{00A250}
    \definecolor{ansi-green-intense}{HTML}{007427}
    \definecolor{ansi-yellow}{HTML}{DDB62B}
    \definecolor{ansi-yellow-intense}{HTML}{B27D12}
    \definecolor{ansi-blue}{HTML}{208FFB}
    \definecolor{ansi-blue-intense}{HTML}{0065CA}
    \definecolor{ansi-magenta}{HTML}{D160C4}
    \definecolor{ansi-magenta-intense}{HTML}{A03196}
    \definecolor{ansi-cyan}{HTML}{60C6C8}
    \definecolor{ansi-cyan-intense}{HTML}{258F8F}
    \definecolor{ansi-white}{HTML}{C5C1B4}
    \definecolor{ansi-white-intense}{HTML}{A1A6B2}

    % commands and environments needed by pandoc snippets
    % extracted from the output of `pandoc -s`
    \providecommand{\tightlist}{%
      \setlength{\itemsep}{0pt}\setlength{\parskip}{0pt}}
    \DefineVerbatimEnvironment{Highlighting}{Verbatim}{commandchars=\\\{\}}
    % Add ',fontsize=\small' for more characters per line
    \newenvironment{Shaded}{}{}
    \newcommand{\KeywordTok}[1]{\textcolor[rgb]{0.00,0.44,0.13}{\textbf{{#1}}}}
    \newcommand{\DataTypeTok}[1]{\textcolor[rgb]{0.56,0.13,0.00}{{#1}}}
    \newcommand{\DecValTok}[1]{\textcolor[rgb]{0.25,0.63,0.44}{{#1}}}
    \newcommand{\BaseNTok}[1]{\textcolor[rgb]{0.25,0.63,0.44}{{#1}}}
    \newcommand{\FloatTok}[1]{\textcolor[rgb]{0.25,0.63,0.44}{{#1}}}
    \newcommand{\CharTok}[1]{\textcolor[rgb]{0.25,0.44,0.63}{{#1}}}
    \newcommand{\StringTok}[1]{\textcolor[rgb]{0.25,0.44,0.63}{{#1}}}
    \newcommand{\CommentTok}[1]{\textcolor[rgb]{0.38,0.63,0.69}{\textit{{#1}}}}
    \newcommand{\OtherTok}[1]{\textcolor[rgb]{0.00,0.44,0.13}{{#1}}}
    \newcommand{\AlertTok}[1]{\textcolor[rgb]{1.00,0.00,0.00}{\textbf{{#1}}}}
    \newcommand{\FunctionTok}[1]{\textcolor[rgb]{0.02,0.16,0.49}{{#1}}}
    \newcommand{\RegionMarkerTok}[1]{{#1}}
    \newcommand{\ErrorTok}[1]{\textcolor[rgb]{1.00,0.00,0.00}{\textbf{{#1}}}}
    \newcommand{\NormalTok}[1]{{#1}}
    
    % Additional commands for more recent versions of Pandoc
    \newcommand{\ConstantTok}[1]{\textcolor[rgb]{0.53,0.00,0.00}{{#1}}}
    \newcommand{\SpecialCharTok}[1]{\textcolor[rgb]{0.25,0.44,0.63}{{#1}}}
    \newcommand{\VerbatimStringTok}[1]{\textcolor[rgb]{0.25,0.44,0.63}{{#1}}}
    \newcommand{\SpecialStringTok}[1]{\textcolor[rgb]{0.73,0.40,0.53}{{#1}}}
    \newcommand{\ImportTok}[1]{{#1}}
    \newcommand{\DocumentationTok}[1]{\textcolor[rgb]{0.73,0.13,0.13}{\textit{{#1}}}}
    \newcommand{\AnnotationTok}[1]{\textcolor[rgb]{0.38,0.63,0.69}{\textbf{\textit{{#1}}}}}
    \newcommand{\CommentVarTok}[1]{\textcolor[rgb]{0.38,0.63,0.69}{\textbf{\textit{{#1}}}}}
    \newcommand{\VariableTok}[1]{\textcolor[rgb]{0.10,0.09,0.49}{{#1}}}
    \newcommand{\ControlFlowTok}[1]{\textcolor[rgb]{0.00,0.44,0.13}{\textbf{{#1}}}}
    \newcommand{\OperatorTok}[1]{\textcolor[rgb]{0.40,0.40,0.40}{{#1}}}
    \newcommand{\BuiltInTok}[1]{{#1}}
    \newcommand{\ExtensionTok}[1]{{#1}}
    \newcommand{\PreprocessorTok}[1]{\textcolor[rgb]{0.74,0.48,0.00}{{#1}}}
    \newcommand{\AttributeTok}[1]{\textcolor[rgb]{0.49,0.56,0.16}{{#1}}}
    \newcommand{\InformationTok}[1]{\textcolor[rgb]{0.38,0.63,0.69}{\textbf{\textit{{#1}}}}}
    \newcommand{\WarningTok}[1]{\textcolor[rgb]{0.38,0.63,0.69}{\textbf{\textit{{#1}}}}}
    
    
    % Define a nice break command that doesn't care if a line doesn't already
    % exist.
    \def\br{\hspace*{\fill} \\* }
    % Math Jax compatability definitions
    \def\gt{>}
    \def\lt{<}
    % Document parameters
    \title{Enumeração exaustiva e outras formas de aproximação\\
    \small{MC102-2018s1-Aula09-180327-takeaway}}
    \author{Arthur J. Catto, PhD}
    \date{27 de março de 2018}
    
    
    

    % Pygments definitions
    
\makeatletter
\def\PY@reset{\let\PY@it=\relax \let\PY@bf=\relax%
    \let\PY@ul=\relax \let\PY@tc=\relax%
    \let\PY@bc=\relax \let\PY@ff=\relax}
\def\PY@tok#1{\csname PY@tok@#1\endcsname}
\def\PY@toks#1+{\ifx\relax#1\empty\else%
    \PY@tok{#1}\expandafter\PY@toks\fi}
\def\PY@do#1{\PY@bc{\PY@tc{\PY@ul{%
    \PY@it{\PY@bf{\PY@ff{#1}}}}}}}
\def\PY#1#2{\PY@reset\PY@toks#1+\relax+\PY@do{#2}}

\expandafter\def\csname PY@tok@w\endcsname{\def\PY@tc##1{\textcolor[rgb]{0.73,0.73,0.73}{##1}}}
\expandafter\def\csname PY@tok@c\endcsname{\let\PY@it=\textit\def\PY@tc##1{\textcolor[rgb]{0.25,0.50,0.50}{##1}}}
\expandafter\def\csname PY@tok@cp\endcsname{\def\PY@tc##1{\textcolor[rgb]{0.74,0.48,0.00}{##1}}}
\expandafter\def\csname PY@tok@k\endcsname{\let\PY@bf=\textbf\def\PY@tc##1{\textcolor[rgb]{0.00,0.50,0.00}{##1}}}
\expandafter\def\csname PY@tok@kp\endcsname{\def\PY@tc##1{\textcolor[rgb]{0.00,0.50,0.00}{##1}}}
\expandafter\def\csname PY@tok@kt\endcsname{\def\PY@tc##1{\textcolor[rgb]{0.69,0.00,0.25}{##1}}}
\expandafter\def\csname PY@tok@o\endcsname{\def\PY@tc##1{\textcolor[rgb]{0.40,0.40,0.40}{##1}}}
\expandafter\def\csname PY@tok@ow\endcsname{\let\PY@bf=\textbf\def\PY@tc##1{\textcolor[rgb]{0.67,0.13,1.00}{##1}}}
\expandafter\def\csname PY@tok@nb\endcsname{\def\PY@tc##1{\textcolor[rgb]{0.00,0.50,0.00}{##1}}}
\expandafter\def\csname PY@tok@nf\endcsname{\def\PY@tc##1{\textcolor[rgb]{0.00,0.00,1.00}{##1}}}
\expandafter\def\csname PY@tok@nc\endcsname{\let\PY@bf=\textbf\def\PY@tc##1{\textcolor[rgb]{0.00,0.00,1.00}{##1}}}
\expandafter\def\csname PY@tok@nn\endcsname{\let\PY@bf=\textbf\def\PY@tc##1{\textcolor[rgb]{0.00,0.00,1.00}{##1}}}
\expandafter\def\csname PY@tok@ne\endcsname{\let\PY@bf=\textbf\def\PY@tc##1{\textcolor[rgb]{0.82,0.25,0.23}{##1}}}
\expandafter\def\csname PY@tok@nv\endcsname{\def\PY@tc##1{\textcolor[rgb]{0.10,0.09,0.49}{##1}}}
\expandafter\def\csname PY@tok@no\endcsname{\def\PY@tc##1{\textcolor[rgb]{0.53,0.00,0.00}{##1}}}
\expandafter\def\csname PY@tok@nl\endcsname{\def\PY@tc##1{\textcolor[rgb]{0.63,0.63,0.00}{##1}}}
\expandafter\def\csname PY@tok@ni\endcsname{\let\PY@bf=\textbf\def\PY@tc##1{\textcolor[rgb]{0.60,0.60,0.60}{##1}}}
\expandafter\def\csname PY@tok@na\endcsname{\def\PY@tc##1{\textcolor[rgb]{0.49,0.56,0.16}{##1}}}
\expandafter\def\csname PY@tok@nt\endcsname{\let\PY@bf=\textbf\def\PY@tc##1{\textcolor[rgb]{0.00,0.50,0.00}{##1}}}
\expandafter\def\csname PY@tok@nd\endcsname{\def\PY@tc##1{\textcolor[rgb]{0.67,0.13,1.00}{##1}}}
\expandafter\def\csname PY@tok@s\endcsname{\def\PY@tc##1{\textcolor[rgb]{0.73,0.13,0.13}{##1}}}
\expandafter\def\csname PY@tok@sd\endcsname{\let\PY@it=\textit\def\PY@tc##1{\textcolor[rgb]{0.73,0.13,0.13}{##1}}}
\expandafter\def\csname PY@tok@si\endcsname{\let\PY@bf=\textbf\def\PY@tc##1{\textcolor[rgb]{0.73,0.40,0.53}{##1}}}
\expandafter\def\csname PY@tok@se\endcsname{\let\PY@bf=\textbf\def\PY@tc##1{\textcolor[rgb]{0.73,0.40,0.13}{##1}}}
\expandafter\def\csname PY@tok@sr\endcsname{\def\PY@tc##1{\textcolor[rgb]{0.73,0.40,0.53}{##1}}}
\expandafter\def\csname PY@tok@ss\endcsname{\def\PY@tc##1{\textcolor[rgb]{0.10,0.09,0.49}{##1}}}
\expandafter\def\csname PY@tok@sx\endcsname{\def\PY@tc##1{\textcolor[rgb]{0.00,0.50,0.00}{##1}}}
\expandafter\def\csname PY@tok@m\endcsname{\def\PY@tc##1{\textcolor[rgb]{0.40,0.40,0.40}{##1}}}
\expandafter\def\csname PY@tok@gh\endcsname{\let\PY@bf=\textbf\def\PY@tc##1{\textcolor[rgb]{0.00,0.00,0.50}{##1}}}
\expandafter\def\csname PY@tok@gu\endcsname{\let\PY@bf=\textbf\def\PY@tc##1{\textcolor[rgb]{0.50,0.00,0.50}{##1}}}
\expandafter\def\csname PY@tok@gd\endcsname{\def\PY@tc##1{\textcolor[rgb]{0.63,0.00,0.00}{##1}}}
\expandafter\def\csname PY@tok@gi\endcsname{\def\PY@tc##1{\textcolor[rgb]{0.00,0.63,0.00}{##1}}}
\expandafter\def\csname PY@tok@gr\endcsname{\def\PY@tc##1{\textcolor[rgb]{1.00,0.00,0.00}{##1}}}
\expandafter\def\csname PY@tok@ge\endcsname{\let\PY@it=\textit}
\expandafter\def\csname PY@tok@gs\endcsname{\let\PY@bf=\textbf}
\expandafter\def\csname PY@tok@gp\endcsname{\let\PY@bf=\textbf\def\PY@tc##1{\textcolor[rgb]{0.00,0.00,0.50}{##1}}}
\expandafter\def\csname PY@tok@go\endcsname{\def\PY@tc##1{\textcolor[rgb]{0.53,0.53,0.53}{##1}}}
\expandafter\def\csname PY@tok@gt\endcsname{\def\PY@tc##1{\textcolor[rgb]{0.00,0.27,0.87}{##1}}}
\expandafter\def\csname PY@tok@err\endcsname{\def\PY@bc##1{\setlength{\fboxsep}{0pt}\fcolorbox[rgb]{1.00,0.00,0.00}{1,1,1}{\strut ##1}}}
\expandafter\def\csname PY@tok@kc\endcsname{\let\PY@bf=\textbf\def\PY@tc##1{\textcolor[rgb]{0.00,0.50,0.00}{##1}}}
\expandafter\def\csname PY@tok@kd\endcsname{\let\PY@bf=\textbf\def\PY@tc##1{\textcolor[rgb]{0.00,0.50,0.00}{##1}}}
\expandafter\def\csname PY@tok@kn\endcsname{\let\PY@bf=\textbf\def\PY@tc##1{\textcolor[rgb]{0.00,0.50,0.00}{##1}}}
\expandafter\def\csname PY@tok@kr\endcsname{\let\PY@bf=\textbf\def\PY@tc##1{\textcolor[rgb]{0.00,0.50,0.00}{##1}}}
\expandafter\def\csname PY@tok@bp\endcsname{\def\PY@tc##1{\textcolor[rgb]{0.00,0.50,0.00}{##1}}}
\expandafter\def\csname PY@tok@fm\endcsname{\def\PY@tc##1{\textcolor[rgb]{0.00,0.00,1.00}{##1}}}
\expandafter\def\csname PY@tok@vc\endcsname{\def\PY@tc##1{\textcolor[rgb]{0.10,0.09,0.49}{##1}}}
\expandafter\def\csname PY@tok@vg\endcsname{\def\PY@tc##1{\textcolor[rgb]{0.10,0.09,0.49}{##1}}}
\expandafter\def\csname PY@tok@vi\endcsname{\def\PY@tc##1{\textcolor[rgb]{0.10,0.09,0.49}{##1}}}
\expandafter\def\csname PY@tok@vm\endcsname{\def\PY@tc##1{\textcolor[rgb]{0.10,0.09,0.49}{##1}}}
\expandafter\def\csname PY@tok@sa\endcsname{\def\PY@tc##1{\textcolor[rgb]{0.73,0.13,0.13}{##1}}}
\expandafter\def\csname PY@tok@sb\endcsname{\def\PY@tc##1{\textcolor[rgb]{0.73,0.13,0.13}{##1}}}
\expandafter\def\csname PY@tok@sc\endcsname{\def\PY@tc##1{\textcolor[rgb]{0.73,0.13,0.13}{##1}}}
\expandafter\def\csname PY@tok@dl\endcsname{\def\PY@tc##1{\textcolor[rgb]{0.73,0.13,0.13}{##1}}}
\expandafter\def\csname PY@tok@s2\endcsname{\def\PY@tc##1{\textcolor[rgb]{0.73,0.13,0.13}{##1}}}
\expandafter\def\csname PY@tok@sh\endcsname{\def\PY@tc##1{\textcolor[rgb]{0.73,0.13,0.13}{##1}}}
\expandafter\def\csname PY@tok@s1\endcsname{\def\PY@tc##1{\textcolor[rgb]{0.73,0.13,0.13}{##1}}}
\expandafter\def\csname PY@tok@mb\endcsname{\def\PY@tc##1{\textcolor[rgb]{0.40,0.40,0.40}{##1}}}
\expandafter\def\csname PY@tok@mf\endcsname{\def\PY@tc##1{\textcolor[rgb]{0.40,0.40,0.40}{##1}}}
\expandafter\def\csname PY@tok@mh\endcsname{\def\PY@tc##1{\textcolor[rgb]{0.40,0.40,0.40}{##1}}}
\expandafter\def\csname PY@tok@mi\endcsname{\def\PY@tc##1{\textcolor[rgb]{0.40,0.40,0.40}{##1}}}
\expandafter\def\csname PY@tok@il\endcsname{\def\PY@tc##1{\textcolor[rgb]{0.40,0.40,0.40}{##1}}}
\expandafter\def\csname PY@tok@mo\endcsname{\def\PY@tc##1{\textcolor[rgb]{0.40,0.40,0.40}{##1}}}
\expandafter\def\csname PY@tok@ch\endcsname{\let\PY@it=\textit\def\PY@tc##1{\textcolor[rgb]{0.25,0.50,0.50}{##1}}}
\expandafter\def\csname PY@tok@cm\endcsname{\let\PY@it=\textit\def\PY@tc##1{\textcolor[rgb]{0.25,0.50,0.50}{##1}}}
\expandafter\def\csname PY@tok@cpf\endcsname{\let\PY@it=\textit\def\PY@tc##1{\textcolor[rgb]{0.25,0.50,0.50}{##1}}}
\expandafter\def\csname PY@tok@c1\endcsname{\let\PY@it=\textit\def\PY@tc##1{\textcolor[rgb]{0.25,0.50,0.50}{##1}}}
\expandafter\def\csname PY@tok@cs\endcsname{\let\PY@it=\textit\def\PY@tc##1{\textcolor[rgb]{0.25,0.50,0.50}{##1}}}

\def\PYZbs{\char`\\}
\def\PYZus{\char`\_}
\def\PYZob{\char`\{}
\def\PYZcb{\char`\}}
\def\PYZca{\char`\^}
\def\PYZam{\char`\&}
\def\PYZlt{\char`\<}
\def\PYZgt{\char`\>}
\def\PYZsh{\char`\#}
\def\PYZpc{\char`\%}
\def\PYZdl{\char`\$}
\def\PYZhy{\char`\-}
\def\PYZsq{\char`\'}
\def\PYZdq{\char`\"}
\def\PYZti{\char`\~}
% for compatibility with earlier versions
\def\PYZat{@}
\def\PYZlb{[}
\def\PYZrb{]}
\makeatother


    % Exact colors from NB
    \definecolor{incolor}{rgb}{0.0, 0.0, 0.5}
    \definecolor{outcolor}{rgb}{0.545, 0.0, 0.0}



    
    % Prevent overflowing lines due to hard-to-break entities
    \sloppy 
    % Setup hyperref package
    \hypersetup{
      breaklinks=true,  % so long urls are correctly broken across lines
      colorlinks=true,
      urlcolor=urlcolor,
      linkcolor=linkcolor,
      citecolor=citecolor,
      }
    % Slightly bigger margins than the latex defaults
    
    \geometry{verbose,tmargin=1in,bmargin=1in,lmargin=1in,rmargin=1in}
    
    

    \begin{document}
    
    
    \maketitle
    
    

    
%    \begin{Verbatim}[commandchars=\\\{\}]
% {\color{incolor}In [{\color{incolor} }]:} \PY{k+kn}{from} \PY{n+nn}{IPython}\PY{n+nn}{.}\PY{n+nn}{core}\PY{n+nn}{.}\PY{n+nn}{interactiveshell} \PY{k}{import} \PY{n}{InteractiveShell}
%        \PY{n}{InteractiveShell}\PY{o}{.}\PY{n}{ast\PYZus{}node\PYZus{}interactivity} \PY{o}{=} \PY{l+s+s2}{\PYZdq{}}\PY{l+s+s2}{all}\PY{l+s+s2}{\PYZdq{}}
%\end{Verbatim}


    \section{Enumeração exaustiva e outras formas de
aproximação}\label{enumerauxe7uxe3o-exaustiva-e-outras-formas-de-aproximauxe7uxe3o}

    Computadores são ótimos para tarefas que exigem grande número de
repetições.

Segundo a Harvard Database of Useful Biological Numbers, uma pessoa
rápida consegue piscar umas 5 vezes por segundo. Nesse mesmo tempo um
processador moderno é capaz de executar pelo menos algumas centenas de
milhões de instruções...

Portanto, vamos botar essas máquinas pra trabalhar...

    \subsection{O método da enumeração
exaustiva}\label{o-muxe9todo-da-enumerauxe7uxe3o-exaustiva}

Chama-se \emph{enumeração exaustiva} um método de solução de problemas
em que nos aproximamos progressivamente de uma solução desejada
examinando todos os possíveis candidatos (ou pelo menos uma parte
considerável deles).

    \subsubsection{\texorpdfstring{Exemplo: \emph{Calcular a raiz cúbica de
um cubo inteiro
perfeito}}{Exemplo: Calcular a raiz cúbica de um cubo inteiro perfeito}}\label{exemplo-calcular-a-raiz-cuxfabica-de-um-cubo-inteiro-perfeito}

    Como o problema só fala de inteiros, este problema pode ser resolvido
por força bruta, examinando exaustivamente todos os possíveis
candidatos.

O conjunto de candidatos pode ser limitado aos inteiros não negativos,
se lembrarmos que quando \(x < 0\), $\sqrt[3]{x} = (-1)
\cdot \sqrt[3]{\lvert x \rvert}$.

Então, vamos lá...

    \begin{Verbatim}[commandchars=\\\{\}]
{\color{incolor}In [{\color{incolor} }]:} \PY{c+c1}{\PYZsh{} Ler o número cuja raiz cúbica se deseja encontrar}
        \PY{n}{x} \PY{o}{=} \PY{n+nb}{int}\PY{p}{(}\PY{n+nb}{input}\PY{p}{(}\PY{l+s+s1}{\PYZsq{}}\PY{l+s+s1}{Digite um número inteiro: }\PY{l+s+s1}{\PYZsq{}}\PY{p}{)}\PY{p}{)}
\end{Verbatim}


    \begin{Verbatim}[commandchars=\\\{\}]
{\color{incolor}In [{\color{incolor} }]:} \PY{c+c1}{\PYZsh{} Encontrar a raiz cúbica de absx por enumeração exaustiva}
        \PY{n}{raiz} \PY{o}{=} 
        \PY{k}{while} \PY{p}{:}
            \PY{n}{raiz} 
\end{Verbatim}


    \begin{Verbatim}[commandchars=\\\{\}]
{\color{incolor}In [{\color{incolor} }]:} \PY{c+c1}{\PYZsh{} Exibir o resultado}
        \PY{k}{if} \PY{p}{:}
            \PY{n+nb}{print}\PY{p}{(}\PY{n}{x}\PY{p}{,} \PY{l+s+s1}{\PYZsq{}}\PY{l+s+s1}{não é um cubo perfeito!}\PY{l+s+s1}{\PYZsq{}}\PY{p}{)}
        \PY{k}{else}\PY{p}{:}
            
            \PY{n+nb}{print}\PY{p}{(}\PY{l+s+s1}{\PYZsq{}}\PY{l+s+s1}{A raiz cúbica de}\PY{l+s+s1}{\PYZsq{}}\PY{p}{,} \PY{n}{x}\PY{p}{,} \PY{l+s+s1}{\PYZsq{}}\PY{l+s+s1}{é}\PY{l+s+s1}{\PYZsq{}}\PY{p}{,} \PY{n}{raiz}\PY{p}{)}
\end{Verbatim}


    \subparagraph{Solução}\label{soluuxe7uxe3o1}

    \begin{Verbatim}[commandchars=\\\{\}]
{\color{incolor}In [{\color{incolor} }]:} \PY{c+c1}{\PYZsh{} Ler o número cuja raiz cúbica se deseja encontrar}
        \PY{n}{x} \PY{o}{=} \PY{n+nb}{int}\PY{p}{(}\PY{n+nb}{input}\PY{p}{(}\PY{l+s+s1}{\PYZsq{}}\PY{l+s+s1}{Digite um número inteiro: }\PY{l+s+s1}{\PYZsq{}}\PY{p}{)}\PY{p}{)}
        \PY{n}{absx} \PY{o}{=} \PY{n+nb}{abs}\PY{p}{(}\PY{n}{x}\PY{p}{)}
\end{Verbatim}


    \begin{Verbatim}[commandchars=\\\{\}]
{\color{incolor}In [{\color{incolor} }]:} \PY{c+c1}{\PYZsh{} Encontrar a raiz cúbica de absx por enumeração exaustiva}
        \PY{n}{raiz} \PY{o}{=} \PY{l+m+mi}{0}
        \PY{k}{while} \PY{n}{raiz} \PY{o}{*}\PY{o}{*} \PY{l+m+mi}{3} \PY{o}{\PYZlt{}} \PY{n}{absx}\PY{p}{:}
            \PY{n}{raiz} \PY{o}{+}\PY{o}{=} \PY{l+m+mi}{1}
\end{Verbatim}


    \begin{Verbatim}[commandchars=\\\{\}]
{\color{incolor}In [{\color{incolor} }]:} \PY{c+c1}{\PYZsh{} Exibir o resultado}
        \PY{k}{if} \PY{n}{raiz} \PY{o}{*}\PY{o}{*} \PY{l+m+mi}{3} \PY{o}{!=} \PY{n}{absx}\PY{p}{:}
            \PY{n+nb}{print}\PY{p}{(}\PY{n}{x}\PY{p}{,} \PY{l+s+s1}{\PYZsq{}}\PY{l+s+s1}{não é um cubo perfeito!}\PY{l+s+s1}{\PYZsq{}}\PY{p}{)}
        \PY{k}{else}\PY{p}{:}
            \PY{k}{if} \PY{n}{x} \PY{o}{\PYZlt{}} \PY{l+m+mi}{0}\PY{p}{:}
                \PY{n}{raiz} \PY{o}{=} \PY{o}{\PYZhy{}}\PY{n}{raiz}
            \PY{n+nb}{print}\PY{p}{(}\PY{l+s+s1}{\PYZsq{}}\PY{l+s+s1}{A raiz cúbica de}\PY{l+s+s1}{\PYZsq{}}\PY{p}{,} \PY{n}{x}\PY{p}{,} \PY{l+s+s1}{\PYZsq{}}\PY{l+s+s1}{é}\PY{l+s+s1}{\PYZsq{}}\PY{p}{,} \PY{n}{raiz}\PY{p}{)}
\end{Verbatim}


    \begin{Verbatim}[commandchars=\\\{\}]
{\color{incolor}In [{\color{incolor} }]:} \PY{l+m+mi}{123456789}\PY{o}{*}\PY{o}{*}\PY{l+m+mi}{3}
\end{Verbatim}


    Seria possível resolver esse problema usando \texttt{for} em vez de
\texttt{while}?

Poderíamos começar com...

    \begin{Verbatim}[commandchars=\\\{\}]
{\color{incolor}In [{\color{incolor} }]:} \PY{c+c1}{\PYZsh{} Encontrar a raiz cúbica de absx por enumeração exaustiva}
        \PY{k}{for} \PY{n}{raiz} \PY{o+ow}{in} \PY{n+nb}{range}\PY{p}{(}\PY{o}{.}\PY{o}{.}\PY{o}{.}\PY{p}{)}\PY{p}{:}
            \PY{o}{.}\PY{o}{.}\PY{o}{.}
\end{Verbatim}


    Qual seria o argumento de \texttt{range}?
    
\begin{itemize}
    \item \texttt{range} precisa
conter a resposta.
    \item Não temos um palpite para isso, mas sabemos que a
raiz cúbica de um inteiro não negativo é sempre menor ou igual a ele.
    \item Portanto, usar \(absx + 1\) como \emph{stop} em \texttt{range} parece
razoável. 
\end{itemize}

Assim, ficamos com...


    \begin{Verbatim}[commandchars=\\\{\}]
{\color{incolor}In [{\color{incolor} }]:} \PY{c+c1}{\PYZsh{} Encontrar a raiz cúbica de absx por enumeração exaustiva}
        \PY{k}{for} \PY{n}{raiz} \PY{o+ow}{in} \PY{n+nb}{range}\PY{p}{(}\PY{n}{absx} \PY{o}{+} \PY{l+m+mi}{1}\PY{p}{)}\PY{p}{:}
            \PY{o}{.}\PY{o}{.}\PY{o}{.}
\end{Verbatim}


    Agora, o que seria uma \emph{suite} aceitável?
    \begin{itemize}
        \item Do jeito em que está, o
\texttt{for} sempre percorrerá a \texttt{range} toda e assim, quase
sempre, passará reto por cima da resposta desejada... 
    \end{itemize}

Como evitar isso?
    \begin{itemize}
        \item É preciso dar um jeito de interromper o \texttt{for} caso a
resposta seja encontrada ou tenhamos certeza de que o problema não tem
solução.
    \end{itemize}


Como fazer isso?

    \begin{itemize}
\tightlist
\item
  Vamos ter que usar um comando condicional...
\end{itemize}

    \begin{Verbatim}[commandchars=\\\{\}]
{\color{incolor}In [{\color{incolor} }]:} \PY{c+c1}{\PYZsh{} Encontrar a raiz cúbica de absx por enumeração exaustiva}
        \PY{k}{for} \PY{n}{raiz} \PY{o+ow}{in} \PY{n+nb}{range}\PY{p}{(}\PY{n}{absx} \PY{o}{+} \PY{l+m+mi}{1}\PY{p}{)}\PY{p}{:}
            \PY{k}{if} \PY{n}{raiz} \PY{o}{*}\PY{o}{*} \PY{l+m+mi}{3} \PY{o}{\PYZgt{}}\PY{o}{=} \PY{n}{x}\PY{p}{:}
                \PY{c+c1}{\PYZsh{} parar o `for`}
\end{Verbatim}


    Para interromper a execução de um \texttt{for}, antes de seu término
normal, usamos um comando \texttt{break}.

    \begin{Verbatim}[commandchars=\\\{\}]
{\color{incolor}In [{\color{incolor} }]:} \PY{c+c1}{\PYZsh{} Encontrar a raiz cúbica de absx por enumeração exaustiva}
        \PY{k}{for} \PY{n}{raiz} \PY{o+ow}{in} \PY{n+nb}{range}\PY{p}{(}\PY{n}{absx} \PY{o}{+} \PY{l+m+mi}{1}\PY{p}{)}\PY{p}{:}
            \PY{k}{if} \PY{n}{raiz} \PY{o}{*}\PY{o}{*} \PY{l+m+mi}{3} \PY{o}{\PYZgt{}}\PY{o}{=} \PY{n}{x}\PY{p}{:}
                \PY{k}{break}
\end{Verbatim}


    E o problema está resolvido...

    Vale lembrar que um \texttt{break} interrompe apenas a execução do
\texttt{for} dentro do qual ele ``realmente'' está.

    \begin{Verbatim}[commandchars=\\\{\}]
{\color{incolor}In [{\color{incolor} }]:} \PY{k}{for} \PY{n}{i} \PY{o+ow}{in} \PY{n+nb}{range}\PY{p}{(}\PY{l+m+mi}{2}\PY{p}{)}\PY{p}{:}
            \PY{n+nb}{print}\PY{p}{(}\PY{l+s+s1}{\PYZsq{}}\PY{l+s+s1}{começando i =}\PY{l+s+s1}{\PYZsq{}}\PY{p}{,} \PY{n}{i}\PY{p}{)}
            \PY{k}{for} \PY{n}{j} \PY{o+ow}{in} \PY{n+nb}{range}\PY{p}{(}\PY{l+m+mi}{2}\PY{p}{)}\PY{p}{:}
                \PY{n+nb}{print}\PY{p}{(}\PY{l+s+s1}{\PYZsq{}}\PY{l+s+s1}{   começando j =}\PY{l+s+s1}{\PYZsq{}}\PY{p}{,} \PY{n}{j}\PY{p}{)}
                \PY{k}{for} \PY{n}{k} \PY{o+ow}{in} \PY{n+nb}{range}\PY{p}{(}\PY{l+m+mi}{2}\PY{p}{)}\PY{p}{:}
                    \PY{n+nb}{print}\PY{p}{(}\PY{l+s+s1}{\PYZsq{}}\PY{l+s+s1}{      começando k =}\PY{l+s+s1}{\PYZsq{}}\PY{p}{,} \PY{n}{k}\PY{p}{)}
                    \PY{n+nb}{print}\PY{p}{(}\PY{l+s+s1}{\PYZsq{}}\PY{l+s+s1}{     break}\PY{l+s+s1}{\PYZsq{}}\PY{p}{)}
                    \PY{k}{break}
                    \PY{n+nb}{print}\PY{p}{(}\PY{l+s+s1}{\PYZsq{}}\PY{l+s+s1}{      terminando k =}\PY{l+s+s1}{\PYZsq{}}\PY{p}{,} \PY{n}{k}\PY{p}{)}
                \PY{n+nb}{print}\PY{p}{(}\PY{l+s+s1}{\PYZsq{}}\PY{l+s+s1}{   terminando j =}\PY{l+s+s1}{\PYZsq{}}\PY{p}{,} \PY{n}{j}\PY{p}{)}
            \PY{n+nb}{print}\PY{p}{(}\PY{l+s+s1}{\PYZsq{}}\PY{l+s+s1}{terminando i =}\PY{l+s+s1}{\PYZsq{}}\PY{p}{,} \PY{n}{i}\PY{p}{)}
        \PY{n+nb}{print}\PY{p}{(}\PY{l+s+s1}{\PYZsq{}}\PY{l+s+s1}{fim do programa}\PY{l+s+s1}{\PYZsq{}}\PY{p}{)}
\end{Verbatim}


    \subsubsection{\texorpdfstring{Exemplo: \emph{Calcular a raiz quadrada
de um número real
não-negativo}}{Exemplo: Calcular a raiz quadrada de um número real não-negativo}}\label{exemplo-calcular-a-raiz-quadrada-de-um-nuxfamero-real-nuxe3o-negativo}

Neste caso, saímos do campo dos inteiros...

Podemos adaptar o exemplo anterior usando um \texttt{for}? Por que?
\begin{itemize}
\tightlist
\item
  Não, porque \texttt{range} não trabalha com \emph{floats}.
\end{itemize}

Mas é possível usar o algoritmo anterior com \texttt{while}.

    Podemos continuar incrementando \(raiz\) de \(1\) em \(1\)? 
    \begin{itemize}
% \tightlist
\item
Provavelmente não... vamos querer uma aproximação melhor que essa... 
\item
Vamos supor que \(10^{-6}\) seja um passo aceitável e vamos atribuir o
nome \(step\) a ele.
\end{itemize}

    As células abaixo contêm nosso exemplo da raiz cúbica. Com algumas
alterações rápidas, nosso programa começará a tomar forma.

Mãos à obra...

    \begin{Verbatim}[commandchars=\\\{\}]
{\color{incolor}In [{\color{incolor} }]:} \PY{c+c1}{\PYZsh{} Ler o número cuja raiz cúbica se deseja encontrar}
        \PY{n}{x} \PY{o}{=} \PY{n+nb}{int}\PY{p}{(}\PY{n+nb}{input}\PY{p}{(}\PY{l+s+s1}{\PYZsq{}}\PY{l+s+s1}{Digite um número inteiro: }\PY{l+s+s1}{\PYZsq{}}\PY{p}{)}\PY{p}{)}
        \PY{n}{absx} \PY{o}{=} \PY{n+nb}{abs}\PY{p}{(}\PY{n}{x}\PY{p}{)}
\end{Verbatim}


    \begin{Verbatim}[commandchars=\\\{\}]
{\color{incolor}In [{\color{incolor} }]:} \PY{c+c1}{\PYZsh{} Encontrar a raiz cúbica de absx por enumeração exaustiva}
        \PY{n}{raiz} \PY{o}{=} \PY{l+m+mi}{0}
        \PY{k}{while} \PY{n}{raiz} \PY{o}{*}\PY{o}{*} \PY{l+m+mi}{3} \PY{o}{\PYZlt{}} \PY{n}{absx}\PY{p}{:}
            \PY{n}{raiz} \PY{o}{+}\PY{o}{=} \PY{l+m+mi}{1}
\end{Verbatim}


    \begin{Verbatim}[commandchars=\\\{\}]
{\color{incolor}In [{\color{incolor} }]:} \PY{c+c1}{\PYZsh{} Exibir o resultado}
        \PY{k}{if} \PY{n}{raiz} \PY{o}{*}\PY{o}{*} \PY{l+m+mi}{3} \PY{o}{!=} \PY{n}{absx}\PY{p}{:}
            \PY{n+nb}{print}\PY{p}{(}\PY{n}{x}\PY{p}{,} \PY{l+s+s1}{\PYZsq{}}\PY{l+s+s1}{não é um cubo perfeito!}\PY{l+s+s1}{\PYZsq{}}\PY{p}{)}
        \PY{k}{else}\PY{p}{:}
            \PY{k}{if} \PY{n}{x} \PY{o}{\PYZlt{}} \PY{l+m+mi}{0}\PY{p}{:}
                \PY{n}{raiz} \PY{o}{=} \PY{o}{\PYZhy{}}\PY{n}{raiz}
            \PY{n+nb}{print}\PY{p}{(}\PY{l+s+s1}{\PYZsq{}}\PY{l+s+s1}{A raiz cúbica de}\PY{l+s+s1}{\PYZsq{}}\PY{p}{,} \PY{n}{x}\PY{p}{,} \PY{l+s+s1}{\PYZsq{}}\PY{l+s+s1}{é}\PY{l+s+s1}{\PYZsq{}}\PY{p}{,} \PY{n}{raiz}\PY{p}{)}
\end{Verbatim}


    \subparagraph{Solução}\label{soluuxe7uxe3o2}

    \begin{Verbatim}[commandchars=\\\{\}]
{\color{incolor}In [{\color{incolor} }]:} \PY{c+c1}{\PYZsh{} Ler o número cuja raiz quadrada se deseja encontrar}
        \PY{n}{x} \PY{o}{=} \PY{n+nb}{float}\PY{p}{(}\PY{n+nb}{input}\PY{p}{(}\PY{l+s+s1}{\PYZsq{}}\PY{l+s+s1}{Digite um número não\PYZhy{}negativo: }\PY{l+s+s1}{\PYZsq{}}\PY{p}{)}\PY{p}{)}
        \PY{n}{step} \PY{o}{=} \PY{l+m+mf}{1e\PYZhy{}6}
\end{Verbatim}


    \begin{Verbatim}[commandchars=\\\{\}]
{\color{incolor}In [{\color{incolor} }]:} \PY{c+c1}{\PYZsh{} Encontrar uma aproximação para a raiz quadrada de x por enumeração exaustiva}
        \PY{n}{raiz} \PY{o}{=} \PY{l+m+mi}{0}
        \PY{k}{while} \PY{n}{raiz} \PY{o}{*}\PY{o}{*} \PY{l+m+mi}{2} \PY{o}{\PYZlt{}}\PY{o}{=} \PY{n}{x}\PY{p}{:}
            \PY{n}{raiz} \PY{o}{+}\PY{o}{=} \PY{n}{step}
        \PY{n+nb}{print}\PY{p}{(}\PY{n}{raiz}\PY{p}{)}
\end{Verbatim}


    \begin{Verbatim}[commandchars=\\\{\}]
{\color{incolor}In [{\color{incolor} }]:} \PY{c+c1}{\PYZsh{} Exibir o resultado}
        \PY{k}{if} \PY{n}{raiz} \PY{o}{*}\PY{o}{*} \PY{l+m+mi}{2} \PY{o}{!=} \PY{n}{x}\PY{p}{:}
            \PY{n+nb}{print}\PY{p}{(}\PY{l+s+s1}{\PYZsq{}}\PY{l+s+s1}{não consegui calcular uma aproximação para a raiz quadrada de}\PY{l+s+s1}{\PYZsq{}}\PY{p}{,} \PY{n}{x}\PY{p}{)}
        \PY{k}{else}\PY{p}{:}
            \PY{n+nb}{print}\PY{p}{(}\PY{n}{raiz}\PY{p}{,} \PY{l+s+s1}{\PYZsq{}}\PY{l+s+s1}{é aproximadamente igual à raiz quadrada de}\PY{l+s+s1}{\PYZsq{}}\PY{p}{,} \PY{n}{x}\PY{p}{)}
\end{Verbatim}


    Vamos testá-lo calculando a raiz quadrada de 25.

    \paragraph{Ops! O que será que
aconteceu?}\label{ops-o-que-seruxe1-que-aconteceu}

Adicionando um \texttt{print(raiz)} no final do cálculo da \texttt{raiz}
vemos que o valor calculado é \texttt{5.000000000344985}, que não está
mal como aproximação...

Por que ele não deu essa resposta? 
\begin{itemize}
\tightlist
\item
Porque a condição do \texttt{if} na
exibição do resultado é \texttt{raiz\ **\ 2\ !=\ x} e o resultado dessa
comparação é \texttt{False}... 
\end{itemize}

Vamos examinar o que está acontecendo
reunindo os trechos essenciais de nossa solução...

    \begin{Verbatim}[commandchars=\\\{\}]
{\color{incolor}In [{\color{incolor} }]:} \PY{n}{x} \PY{o}{=} \PY{l+m+mi}{25}
        \PY{n}{step} \PY{o}{=} \PY{l+m+mf}{1e\PYZhy{}6}
        \PY{n}{raiz} \PY{o}{=} \PY{l+m+mi}{0}
        \PY{k}{while} \PY{n}{raiz} \PY{o}{*}\PY{o}{*} \PY{l+m+mi}{2} \PY{o}{\PYZlt{}}\PY{o}{=} \PY{n}{x}\PY{p}{:}
            \PY{n}{raiz} \PY{o}{+}\PY{o}{=} \PY{n}{step}
        \PY{n+nb}{print}\PY{p}{(}\PY{n}{x}\PY{p}{,} \PY{n}{raiz}\PY{p}{,} \PY{n}{raiz} \PY{o}{*}\PY{o}{*} \PY{l+m+mi}{2}\PY{p}{,} \PY{p}{(}\PY{n}{raiz} \PY{o}{*}\PY{o}{*} \PY{l+m+mi}{2} \PY{o}{!=} \PY{n}{x}\PY{p}{)}\PY{p}{)}
\end{Verbatim}


    Isso acontece sempre que se usam \emph{floats} para representar
\emph{reais} e decorre de como \emph{floats} são representados num
computador (ou mesmo numa folha de papel).

    \paragraph{\texorpdfstring{Uma ligeira digressão para falar de
\emph{floats}.}{Uma ligeira digressão para falar de floats.}}\label{uma-ligeira-digressuxe3o-para-falar-de-floats}

    Numa folha de papel ou num computador, somente conseguimos representar
números com uma quantidade limitada de algarismos significativos. Por
maior que seja a quantidade de algarismos usados, isso representará
quase nada do espaço infinito dos \emph{reais}.

Para conseguir expandir o intervalo que seria coberto pelos
\emph{floats}, sacrificamos a precisão e adotamos a \emph{notação
exponencial}. Vamos ver como.

Na \emph{notação exponencial}, um número é representado como um inteiro
com um certo número de algarismos significativos multiplicado por 10
elevado a um certo expoente. Quando estamos fazendo cálculos à mão, o
número de algarismos significativos usados dependerá de nossa escolha.
Para facilitar a leitura, suponha que sejam 6. Por exemplo,...

\begin{align*}
1.25 &= 125000 \cdot 10^{-5} \\
3.1416 &= 314160 \cdot 10^{-5} \\
123.456 &= 123456 \cdot 10^{-3}
\end{align*}

    Com \(6\) algarismos significativos e em notação decimal, o maior número
que se consegue representar é \(999999\). Em notação exponencial, graças
ao fator \(10^e\), podemos ir muito além. O único limite será imposto
pela faixa aceitável para o expoente \(e\). Por exemplo,...

\begin{align*}
999999 &= 999999 \cdot 10^{0} \\
999999000 &= 999999 \cdot 10^{3} \\
999999000000 &= 999999 \cdot 10^{6}
\end{align*}

    Suponha agora que \(a = 999999000000\) e \(b = 999999999999\).

Como eles seriam representados?

\begin{align*}
a = 999999000000 &= 999999 \cdot 10^{6} \\
b = 999999999999 &= 999999 \cdot 10^{6}
\end{align*}

E se agora calcularmos \(b - a\)? 
\begin{itemize}
\tightlist
\item
O resultado será
\(000000 \cdot 10^0\), quando deveria ser \(999999 \cdot 10^0\).
\end{itemize}

    Você vê a perda de precisão?

Num computador acontece exatamente a mesma coisa. Apenas, como o
computador trabalha na base 2 e não na base 10, os valores críticos
serão outros. Por exemplo,...

    \begin{Verbatim}[commandchars=\\\{\}]
{\color{incolor}In [{\color{incolor} }]:} \PY{n}{a} \PY{o}{=} \PY{l+m+mi}{2}\PY{o}{*}\PY{o}{*}\PY{l+m+mi}{72}
        \PY{n}{b} \PY{o}{=} \PY{n}{a} \PY{o}{+} \PY{l+m+mi}{524288}
        \PY{n}{fa} \PY{o}{=} \PY{n+nb}{float}\PY{p}{(}\PY{n}{a}\PY{p}{)}
        \PY{n}{fb} \PY{o}{=} \PY{n+nb}{float}\PY{p}{(}\PY{n}{b}\PY{p}{)}
        \PY{n+nb}{print}\PY{p}{(}\PY{n+nb}{format}\PY{p}{(}\PY{n}{a}\PY{p}{,} \PY{l+s+s2}{\PYZdq{}}\PY{l+s+s2}{24d}\PY{l+s+s2}{\PYZdq{}}\PY{p}{)}\PY{p}{,} \PY{n+nb}{format}\PY{p}{(}\PY{n}{b}\PY{p}{,} \PY{l+s+s2}{\PYZdq{}}\PY{l+s+s2}{28d}\PY{l+s+s2}{\PYZdq{}}\PY{p}{)}\PY{p}{,} \PY{n+nb}{format}\PY{p}{(}\PY{n}{b} \PY{o}{\PYZhy{}} \PY{n}{a}\PY{p}{,} \PY{l+s+s2}{\PYZdq{}}\PY{l+s+s2}{32d}\PY{l+s+s2}{\PYZdq{}}\PY{p}{)}\PY{p}{)}
        \PY{n+nb}{print}\PY{p}{(}\PY{n+nb}{format}\PY{p}{(}\PY{n}{fa}\PY{p}{,} \PY{l+s+s2}{\PYZdq{}}\PY{l+s+s2}{28.21e}\PY{l+s+s2}{\PYZdq{}}\PY{p}{)}\PY{p}{,} \PY{n+nb}{format}\PY{p}{(}\PY{n}{fb}\PY{p}{,} \PY{l+s+s2}{\PYZdq{}}\PY{l+s+s2}{28.21e}\PY{l+s+s2}{\PYZdq{}}\PY{p}{)}\PY{p}{,} \PY{n+nb}{format}\PY{p}{(}\PY{n}{fb} \PY{o}{\PYZhy{}} \PY{n}{fa}\PY{p}{,} \PY{l+s+s2}{\PYZdq{}}\PY{l+s+s2}{28.21e}\PY{l+s+s2}{\PYZdq{}}\PY{p}{)}\PY{p}{)}
\end{Verbatim}


    Não são apenas grandes números que sofrem com a perda de precisão.

Na notação exponencial decimal com 6 algarismos significativos
\(\frac{1}{3}\) é representado como \(333333 \cdot 10^{-6} \nonumber\).

Assim, quando somamos \(\frac{1}{3} + \frac{1}{3} + \frac{1}{3}\),
obtemos \(999999 \cdot 10^{-6} \nonumber\) que é diferente de
\(1\ (100000 \cdot 10^{-5})\)

    Como no caso dos grandes números, esse problema também ocorre num
computador digital, embora com outros valores por causa da mudança da
base. Por exemplo, \(\frac{1}{10}\) não tem representação exata na base
\(2\), assim como \(\frac{1}{3}\) não tem na base \(10\)...

Assim, se somarmos \(0.1\) dez vezes, o resultado obtido não será
rigorosamente igual a \(1.0\)...

    \begin{Verbatim}[commandchars=\\\{\}]
{\color{incolor}In [{\color{incolor} }]:} \PY{n}{tot} \PY{o}{=} \PY{l+m+mf}{0.0}
        \PY{k}{for} \PY{n}{i} \PY{o+ow}{in} \PY{n+nb}{range}\PY{p}{(}\PY{l+m+mi}{10}\PY{p}{)}\PY{p}{:}
            \PY{n}{tot} \PY{o}{+}\PY{o}{=} \PY{l+m+mf}{0.1}
        \PY{n+nb}{print}\PY{p}{(}\PY{l+m+mf}{1.0}\PY{p}{,} \PY{n}{tot}\PY{p}{,} \PY{n}{tot} \PY{o}{==} \PY{l+m+mf}{1.0}\PY{p}{,} \PY{l+m+mi}{1} \PY{o}{\PYZhy{}} \PY{n}{tot}\PY{p}{)}
\end{Verbatim}


    A perda de precisão em cálculos relativamente simples pode ser muito
significativa. Suponha que queiramos calcular

\begin{align*}\frac{(x + a)^2 - 2 x a - a^2}{x^2}\end{align*}

O resultado esperado dessa expressão é \(1\), quaisquer que sejam \(x\)
e \(a\), desde que \(x \ne 0\). Vejamos o que acontece quando criamos um
\emph{script} Python para fazer esse cálculo usando vários valores de
\(a\) e \(x\)...

    \begin{Verbatim}[commandchars=\\\{\}]
{\color{incolor}In [{\color{incolor} }]:} \PY{n+nb}{print}\PY{p}{(}\PY{l+m+mi}{7} \PY{o}{*} \PY{l+s+s1}{\PYZsq{}}\PY{l+s+s1}{ }\PY{l+s+s1}{\PYZsq{}} \PY{o}{+} \PY{l+s+s1}{\PYZsq{}}\PY{l+s+s1}{x}\PY{l+s+s1}{\PYZsq{}} \PY{o}{+} \PY{l+m+mi}{15} \PY{o}{*} \PY{l+s+s1}{\PYZsq{}}\PY{l+s+s1}{ }\PY{l+s+s1}{\PYZsq{}} \PY{o}{+} \PY{l+s+s1}{\PYZsq{}}\PY{l+s+s1}{a}\PY{l+s+s1}{\PYZsq{}} \PY{o}{+} \PY{l+m+mi}{16} \PY{o}{*} \PY{l+s+s1}{\PYZsq{}}\PY{l+s+s1}{ }\PY{l+s+s1}{\PYZsq{}} \PY{o}{+} \PY{l+s+s1}{\PYZsq{}}\PY{l+s+s1}{f(x, a)}\PY{l+s+s1}{\PYZsq{}} \PY{o}{+} \PY{l+m+mi}{16} \PY{o}{*} \PY{l+s+s1}{\PYZsq{}}\PY{l+s+s1}{ }\PY{l+s+s1}{\PYZsq{}} \PY{o}{+} \PY{l+s+s1}{\PYZsq{}}\PY{l+s+s1}{e(x, a)}\PY{l+s+s1}{\PYZsq{}}\PY{p}{)}
        \PY{n+nb}{print}\PY{p}{(}\PY{l+m+mi}{14} \PY{o}{*} \PY{l+s+s1}{\PYZsq{}}\PY{l+s+s1}{\PYZhy{}}\PY{l+s+s1}{\PYZsq{}} \PY{o}{+} \PY{l+s+s1}{\PYZsq{}}\PY{l+s+s1}{  }\PY{l+s+s1}{\PYZsq{}} \PY{o}{+} \PY{l+m+mi}{14} \PY{o}{*} \PY{l+s+s1}{\PYZsq{}}\PY{l+s+s1}{\PYZhy{}}\PY{l+s+s1}{\PYZsq{}} \PY{o}{+} \PY{l+s+s1}{\PYZsq{}}\PY{l+s+s1}{  }\PY{l+s+s1}{\PYZsq{}} \PY{o}{+} \PY{l+m+mi}{21} \PY{o}{*} \PY{l+s+s1}{\PYZsq{}}\PY{l+s+s1}{\PYZhy{}}\PY{l+s+s1}{\PYZsq{}} \PY{o}{+} \PY{l+s+s1}{\PYZsq{}}\PY{l+s+s1}{  }\PY{l+s+s1}{\PYZsq{}} \PY{o}{+} \PY{l+m+mi}{21} \PY{o}{*} \PY{l+s+s1}{\PYZsq{}}\PY{l+s+s1}{\PYZhy{}}\PY{l+s+s1}{\PYZsq{}}\PY{p}{)}
        \PY{k}{for} \PY{n}{p} \PY{o+ow}{in} \PY{n+nb}{range} \PY{p}{(}\PY{l+m+mi}{6}\PY{p}{)}\PY{p}{:}
            \PY{n}{a} \PY{o}{=} \PY{l+m+mi}{10} \PY{o}{*}\PY{o}{*} \PY{n}{p}
            \PY{n}{x} \PY{o}{=} \PY{l+m+mi}{10} \PY{o}{*}\PY{o}{*} \PY{o}{\PYZhy{}}\PY{p}{(}\PY{n}{p} \PY{o}{+} \PY{l+m+mi}{1}\PY{p}{)}
            \PY{n}{f} \PY{o}{=} \PY{p}{(}\PY{p}{(}\PY{n}{x} \PY{o}{+} \PY{n}{a}\PY{p}{)} \PY{o}{*}\PY{o}{*} \PY{l+m+mi}{2} \PY{o}{\PYZhy{}} \PY{l+m+mi}{2} \PY{o}{*} \PY{n}{x} \PY{o}{*} \PY{n}{a} \PY{o}{\PYZhy{}} \PY{n}{a} \PY{o}{*}\PY{o}{*} \PY{l+m+mi}{2}\PY{p}{)} \PY{o}{/} \PY{p}{(}\PY{n}{x} \PY{o}{*}\PY{o}{*} \PY{l+m+mi}{2}\PY{p}{)}
            \PY{n}{e} \PY{o}{=} \PY{p}{(}\PY{p}{(}\PY{n}{a} \PY{o}{+} \PY{n}{x}\PY{p}{)} \PY{o}{*}\PY{o}{*} \PY{l+m+mi}{2} \PY{o}{\PYZhy{}} \PY{n}{a} \PY{o}{*}\PY{o}{*} \PY{l+m+mi}{2} \PY{o}{\PYZhy{}} \PY{l+m+mi}{2} \PY{o}{*} \PY{n}{a} \PY{o}{*} \PY{n}{x}\PY{p}{)} \PY{o}{/} \PY{p}{(}\PY{n}{x} \PY{o}{*}\PY{o}{*} \PY{l+m+mi}{2}\PY{p}{)}
            \PY{n+nb}{print}\PY{p}{(}\PY{n}{f}\PY{l+s+s1}{\PYZsq{}}\PY{l+s+si}{\PYZob{}x:14.8e\PYZcb{}}\PY{l+s+s1}{  }\PY{l+s+si}{\PYZob{}a:14.8e\PYZcb{}}\PY{l+s+s1}{  }\PY{l+s+si}{\PYZob{}f:21.14e\PYZcb{}}\PY{l+s+s1}{  }\PY{l+s+si}{\PYZob{}e:21.14e\PYZcb{}}\PY{l+s+s1}{\PYZsq{}}\PY{p}{)}
\end{Verbatim}


    Todos os valores nas colunas \texttt{f(x,\ a)} e \texttt{e(x,\ a)}
deveriam ser "iguais" a \(1.0\), mas vários estão longe disso.

Esse não é um privilégio de Python, mas sim uma consequência da forma
como \emph{floats} estão implementados nos processadores usados nos
nossos computadores. Toda vez que você operar com números muito díspares
(neste caso \(a^2\) é muito maior que \(x^2\)) haverá o risco de perda
grave de precisão. Se você fizer o mesmo cálculo em outras plataformas
(inclusive Excel) vai encontrar o mesmo resultado.

    \subsubsection{De volta ao nosso
exemplo...}\label{de-volta-ao-nosso-exemplo...}

    A discussão sobre \emph{floats} deixou claro o perigo de fazermos testes
de igualdade com eles. A saída é usar \emph{testes de proximidade}. Um
\emph{teste de proximidade} compara dois valores e vê se a distância
entre eles está dentro de uma certa tolerância \(\varepsilon\)
(\emph{epsilon}) pré-definida.

Vamos fazer isso no nosso algoritmo...

    \begin{Verbatim}[commandchars=\\\{\}]
{\color{incolor}In [{\color{incolor} }]:} \PY{c+c1}{\PYZsh{} Encontrar uma aproximação para a raiz quadrada de x por enumeração exaustiva}
        \PY{n}{raiz} \PY{o}{=} \PY{l+m+mi}{0}
        \PY{k}{while} \PY{n+nb}{abs}\PY{p}{(}\PY{n}{raiz} \PY{o}{*}\PY{o}{*} \PY{l+m+mi}{2} \PY{o}{\PYZhy{}} \PY{n}{x}\PY{p}{)} \PY{o}{\PYZgt{}} \PY{n}{epsilon}\PY{p}{:}
            \PY{n}{raiz} \PY{o}{+}\PY{o}{=} \PY{n}{step}
\end{Verbatim}


    Vamos reunir os pedaços e testar o resultado... mas antes disso
precisamos escolher um valor razoável para \emph{epsilon}. O que você
acha?

    \begin{Verbatim}[commandchars=\\\{\}]
{\color{incolor}In [{\color{incolor} }]:} \PY{c+c1}{\PYZsh{} Ler o número cuja raiz quadrada se deseja encontrar}
        \PY{n}{x} \PY{o}{=} \PY{n+nb}{float}\PY{p}{(}\PY{n+nb}{input}\PY{p}{(}\PY{l+s+s1}{\PYZsq{}}\PY{l+s+s1}{Digite um número não\PYZhy{}negativo: }\PY{l+s+s1}{\PYZsq{}}\PY{p}{)}\PY{p}{)}
        \PY{n}{step} \PY{o}{=} \PY{l+m+mf}{1e\PYZhy{}6}
        \PY{n}{epsilon} \PY{o}{=} \PY{l+m+mf}{1e\PYZhy{}4}
\end{Verbatim}


    \begin{Verbatim}[commandchars=\\\{\}]
{\color{incolor}In [{\color{incolor} }]:} \PY{c+c1}{\PYZsh{} Encontrar uma aproximação para a raiz quadrada de x por enumeração exaustiva}
        \PY{n}{raiz} \PY{o}{=} \PY{l+m+mi}{0}
        \PY{k}{while} \PY{n+nb}{abs}\PY{p}{(}\PY{n}{raiz} \PY{o}{*}\PY{o}{*} \PY{l+m+mi}{2} \PY{o}{\PYZhy{}} \PY{n}{x}\PY{p}{)} \PY{o}{\PYZgt{}} \PY{n}{epsilon}\PY{p}{:}
            \PY{n}{raiz} \PY{o}{+}\PY{o}{=} \PY{n}{step}
\end{Verbatim}


    \begin{Verbatim}[commandchars=\\\{\}]
{\color{incolor}In [{\color{incolor} }]:} \PY{c+c1}{\PYZsh{} Exibir o resultado}
        \PY{n+nb}{print}\PY{p}{(}\PY{n}{raiz}\PY{p}{,} \PY{l+s+s1}{\PYZsq{}}\PY{l+s+s1}{é aproximadamente igual à raiz quadrada de}\PY{l+s+s1}{\PYZsq{}}\PY{p}{,} \PY{n}{x}\PY{p}{)}
\end{Verbatim}


    O que acontece com nosso programa se o número dado for muito grande? Por
exemplo, 1524157875019052100?

\paragraph{Cansamos de esperar?} Esse inteiro é o quadrado de 1234567890. Como
partimos de zero e estamos usando um \emph{step} de \(10^{-6}\), serão
necessários mais de \(10^{15}\) ciclos do \texttt{while} para chegar à
resposta. Isso é muita coisa, mesmo para um computador muito rápido.

Por outro lado, se usarmos um \emph{step} maior, nosso programa poderá
deixar de funcionar para números pequenos. O que fazer?

    Uma saída seria usar um \emph{step} variável, que pudesse ir sendo
refinado à medida que nos aproximássemos da solução.

    \subsection{O método da bisseção}\label{o-muxe9todo-da-bisseuxe7uxe3o}

O \emph{método da bisseção} é um método de aproximação com \emph{step}
variável, muito rápido e muito usado na solução de diversos problemas
reais.

Suponha que sejamos capazes de definir um intervalo \([esq..dir]\) no
qual, com certeza, a resposta está contida. Vamos escolher como
candidato o valor médio desse intervalo e associá-lo à variável
\(raiz\).

Agora, duas coisas podem acontecer:
\begin{itemize}
% \tightlist
\item
\(\lvert\, raiz^2 - x \,\rvert \le \varepsilon \rightarrow\) nosso
problema está resolvido, ou, 
\item
\(\lvert\, raiz^2 - x \,\rvert \gt \varepsilon \rightarrow\) é preciso
encontrar um novo candidato. Nesse caso... 
    \begin{itemize}
        % \tightlist
        \item
Se \(raiz^2 < x\), o
candidato era pequeno demais. Portanto, a resposta deverá estar no
intervalo \([raiz..dir]\) e um novo candidato deverá ser procurado
dentro dele. 
        \item
Se \(raiz^2 > x\), o candidato era grande demais.
Portanto, a resposta deverá estar no intervalo \([esq..raiz]\) e um novo
candidato deverá ser procurado dentro dele.
    \end{itemize}
\end{itemize}

Note que, a cada passo, a largura do intervalo onde a resposta
certamente se encontra é reduzida pela metade, o que explica o nome do
método e a velocidade com que o algoritmo se aproxima da solução.

    Vamos adaptar nosso algoritmo que emprega o método de enumeração
exaustiva para que ele adote o método da bisseção.

    \begin{Verbatim}[commandchars=\\\{\}]
{\color{incolor}In [{\color{incolor} }]:} \PY{c+c1}{\PYZsh{} Ler o número cuja raiz quadrada se deseja encontrar}
        \PY{n}{x} \PY{o}{=} \PY{n+nb}{float}\PY{p}{(}\PY{n+nb}{input}\PY{p}{(}\PY{l+s+s1}{\PYZsq{}}\PY{l+s+s1}{Digite um número não\PYZhy{}negativo: }\PY{l+s+s1}{\PYZsq{}}\PY{p}{)}\PY{p}{)}
        \PY{n}{epsilon} \PY{o}{=} \PY{l+m+mf}{1e\PYZhy{}8}
\end{Verbatim}


    \begin{Verbatim}[commandchars=\\\{\}]
{\color{incolor}In [{\color{incolor} }]:} \PY{c+c1}{\PYZsh{} Encontrar uma aproximação para a raiz quadrada de x por bisseção}
        
        \PY{c+c1}{\PYZsh{} Definição do intervalo de busca inicial e do primeiro candidato}
        \PY{k}{if} \PY{n}{x} \PY{o}{\PYZgt{}} \PY{l+m+mi}{1}\PY{p}{:}
            \PY{n}{lim\PYZus{}esq}\PY{p}{,} \PY{n}{lim\PYZus{}dir} \PY{o}{=} \PY{l+m+mf}{1.0}\PY{p}{,} \PY{n}{x}
        \PY{k}{else}\PY{p}{:}
            \PY{n}{lim\PYZus{}esq}\PY{p}{,} \PY{n}{lim\PYZus{}dir} \PY{o}{=} \PY{n}{x}\PY{p}{,} \PY{l+m+mf}{1.0}
        
        \PY{n}{raiz} \PY{o}{=} \PY{p}{(}\PY{n}{lim\PYZus{}esq} \PY{o}{+} \PY{n}{lim\PYZus{}dir}\PY{p}{)} \PY{o}{/} \PY{l+m+mi}{2}
        
        \PY{c+c1}{\PYZsh{} refinamento progressivo da aproximação}
        \PY{k}{while} \PY{n+nb}{abs}\PY{p}{(}\PY{n}{raiz} \PY{o}{*}\PY{o}{*} \PY{l+m+mi}{2} \PY{o}{\PYZhy{}} \PY{n}{x}\PY{p}{)} \PY{o}{\PYZgt{}} \PY{n}{epsilon}\PY{p}{:}
            \PY{k}{if} \PY{n}{raiz} \PY{o}{*}\PY{o}{*} \PY{l+m+mi}{2} \PY{o}{\PYZlt{}} \PY{n}{x}\PY{p}{:}
                \PY{n}{lim\PYZus{}esq} \PY{o}{=} \PY{n}{raiz}
            \PY{k}{else}\PY{p}{:}
                \PY{n}{lim\PYZus{}dir} \PY{o}{=} \PY{n}{raiz}
            \PY{n}{raiz} \PY{o}{=} \PY{p}{(}\PY{n}{lim\PYZus{}esq} \PY{o}{+} \PY{n}{lim\PYZus{}dir}\PY{p}{)} \PY{o}{/} \PY{l+m+mi}{2}
\end{Verbatim}


    \begin{Verbatim}[commandchars=\\\{\}]
{\color{incolor}In [{\color{incolor} }]:} \PY{c+c1}{\PYZsh{} Exibir o resultado}
        \PY{n+nb}{print}\PY{p}{(}\PY{n}{raiz}\PY{p}{,} \PY{l+s+s1}{\PYZsq{}}\PY{l+s+s1}{é a raiz quadrada de}\PY{l+s+s1}{\PYZsq{}}\PY{p}{,} \PY{n}{x}\PY{p}{,} \PY{l+s+s1}{\PYZsq{}}\PY{l+s+s1}{a menos de}\PY{l+s+s1}{\PYZsq{}}\PY{p}{,} \PY{n}{epsilon}\PY{p}{)}
\end{Verbatim}


    Vamos testar nosso programa com os valores usados anteriormente. Será
que ele consegue calcular um aproximação da raiz quadrada de
1524157875019052100?

    Lembra-se de que nesse caso o método de busca exaustiva necessitaria de
mais do que \(10^{15}\) ciclos do \texttt{while} para chegar à resposta?
De quantos ciclos será que o método da bisseção necessitou?

Vamos incluir um contador, só por curiosidade...

    \begin{Verbatim}[commandchars=\\\{\}]
{\color{incolor}In [{\color{incolor} }]:} \PY{c+c1}{\PYZsh{} Encontrar uma aproximação para a raiz quadrada de x por bisseção}
        \PY{c+c1}{\PYZsh{} Definição do intervalo de busca inicial}
        \PY{k}{if} \PY{n}{x} \PY{o}{\PYZgt{}} \PY{l+m+mi}{1}\PY{p}{:}
            \PY{n}{lim\PYZus{}esq}\PY{p}{,} \PY{n}{lim\PYZus{}dir} \PY{o}{=} \PY{l+m+mf}{1.0}\PY{p}{,} \PY{n}{x}
        \PY{k}{else}\PY{p}{:}
            \PY{n}{lim\PYZus{}esq}\PY{p}{,} \PY{n}{lim\PYZus{}dir} \PY{o}{=} \PY{n}{x}\PY{p}{,} \PY{l+m+mf}{1.0}
        \PY{c+c1}{\PYZsh{} O candidato é o ponto médio do intervalo}
        \PY{n}{raiz} \PY{o}{=} \PY{p}{(}\PY{n}{lim\PYZus{}esq} \PY{o}{+} \PY{n}{lim\PYZus{}dir}\PY{p}{)} \PY{o}{/} \PY{l+m+mi}{2}
        
        \PY{n}{ciclos} \PY{o}{=} \PY{l+m+mi}{0}
        \PY{k}{while} \PY{n+nb}{abs}\PY{p}{(}\PY{n}{raiz} \PY{o}{*}\PY{o}{*} \PY{l+m+mi}{2} \PY{o}{\PYZhy{}} \PY{n}{x}\PY{p}{)} \PY{o}{\PYZgt{}} \PY{n}{epsilon}\PY{p}{:}
            \PY{k}{if} \PY{n}{raiz} \PY{o}{*}\PY{o}{*} \PY{l+m+mi}{2} \PY{o}{\PYZlt{}} \PY{n}{x}\PY{p}{:}
                \PY{n}{lim\PYZus{}esq} \PY{o}{=} \PY{n}{raiz}
            \PY{k}{else}\PY{p}{:}
                \PY{n}{lim\PYZus{}dir} \PY{o}{=} \PY{n}{raiz}
            \PY{n}{raiz} \PY{o}{=} \PY{p}{(}\PY{n}{lim\PYZus{}esq} \PY{o}{+} \PY{n}{lim\PYZus{}dir}\PY{p}{)} \PY{o}{/} \PY{l+m+mi}{2}
            \PY{n}{ciclos} \PY{o}{+}\PY{o}{=} \PY{l+m+mi}{1}
        \PY{n+nb}{print}\PY{p}{(}\PY{l+s+s1}{\PYZsq{}}\PY{l+s+s1}{número de ciclos necessários =}\PY{l+s+s1}{\PYZsq{}}\PY{p}{,} \PY{n}{ciclos}\PY{p}{)}
\end{Verbatim}


    \subsection{O método de Newton (ou
Newton-Raphson)}\label{o-muxe9todo-de-newton-ou-newton-raphson}

Newton criou um método que é usualmente adotado para encontrar as raízes
reais de muitas funções e que se aplica também ao nosso exemplo. Raphson
propôs uma ideia semelhante mais ou menos ao mesmo tempo e, por isso, o
método é citado com os dois nomes.

No nosso caso, dado um número real não-negativo \(x\), queremos
encontrar \(r\) tal que \(r^2 = x\). Passando \(x\) para o lado
esquerdo, vemos que esse problema é o mesmo que encontrar a raiz do
polinômio \(p(r) = r^2 - x\), isto é, um valor \(raiz\) tal que
\(p(raiz) = 0\).

Newton demonstrou que se \(raiz\) é uma aproximação da resposta
desejada, \(raiz - \dfrac{p(raiz)}{\dot p(raiz)}\) é uma aproximação
melhor ainda.

Vamos fazer alguns pequenos ajustes no nosso algoritmo para que ele
implemente esse modelo.

    \begin{Verbatim}[commandchars=\\\{\}]
{\color{incolor}In [{\color{incolor} }]:} \PY{c+c1}{\PYZsh{} Ler o número cuja raiz quadrada se deseja encontrar}
        \PY{n}{x} \PY{o}{=} \PY{n+nb}{float}\PY{p}{(}\PY{n+nb}{input}\PY{p}{(}\PY{l+s+s1}{\PYZsq{}}\PY{l+s+s1}{Digite um número não\PYZhy{}negativo: }\PY{l+s+s1}{\PYZsq{}}\PY{p}{)}\PY{p}{)}
        \PY{n}{epsilon} \PY{o}{=} \PY{l+m+mf}{1e\PYZhy{}8}
\end{Verbatim}


    \begin{Verbatim}[commandchars=\\\{\}]
{\color{incolor}In [{\color{incolor} }]:} \PY{c+c1}{\PYZsh{} Encontrar uma aproximação para a raiz quadrada de x por Newton\PYZhy{}Raphson}
        \PY{c+c1}{\PYZsh{} Definição do candidato inicial}
        \PY{n}{raiz} \PY{o}{=} \PY{n}{x} \PY{o}{/} \PY{l+m+mf}{2.0}
        
        \PY{n}{ciclos} \PY{o}{=} \PY{l+m+mi}{0}
        \PY{k}{while} \PY{n+nb}{abs}\PY{p}{(}\PY{n}{raiz} \PY{o}{*}\PY{o}{*} \PY{l+m+mi}{2} \PY{o}{\PYZhy{}} \PY{n}{x}\PY{p}{)} \PY{o}{\PYZgt{}} \PY{n}{epsilon}\PY{p}{:}
            \PY{n}{p} \PY{o}{=} \PY{n}{raiz} \PY{o}{*}\PY{o}{*} \PY{l+m+mi}{2} \PY{o}{\PYZhy{}} \PY{n}{x}  \PY{c+c1}{\PYZsh{} este é o polinômio calculado em \PYZdq{}raiz\PYZdq{}}
            \PY{n}{dp} \PY{o}{=} \PY{l+m+mi}{2} \PY{o}{*} \PY{n}{raiz}      \PY{c+c1}{\PYZsh{} esta é a derivada calculada em \PYZdq{}raiz\PYZdq{}}
            \PY{n}{raiz} \PY{o}{\PYZhy{}}\PY{o}{=} \PY{n}{p} \PY{o}{/} \PY{n}{dp}
            \PY{n}{ciclos} \PY{o}{+}\PY{o}{=} \PY{l+m+mi}{1}
        \PY{n+nb}{print}\PY{p}{(}\PY{l+s+s1}{\PYZsq{}}\PY{l+s+s1}{número de ciclos necessários =}\PY{l+s+s1}{\PYZsq{}}\PY{p}{,} \PY{n}{ciclos}\PY{p}{)}
\end{Verbatim}


    \begin{Verbatim}[commandchars=\\\{\}]
{\color{incolor}In [{\color{incolor} }]:} \PY{c+c1}{\PYZsh{} Exibir o resultado}
        \PY{n+nb}{print}\PY{p}{(}\PY{n}{raiz}\PY{p}{,} \PY{l+s+s1}{\PYZsq{}}\PY{l+s+s1}{é a raiz quadrada de}\PY{l+s+s1}{\PYZsq{}}\PY{p}{,} \PY{n}{x}\PY{p}{,} \PY{l+s+s1}{\PYZsq{}}\PY{l+s+s1}{a menos de}\PY{l+s+s1}{\PYZsq{}}\PY{p}{,} \PY{n}{epsilon}\PY{p}{)}
\end{Verbatim}


    \emph{Newton-Raphson} precisou de menos do que a metade dos ciclos
necessários para o método da \emph{bisseção}, que já era muito mais
rápido e geral do que a \emph{enumeração exaustiva}.

    \subsubsection{\texorpdfstring{Exercício: \emph{Encontrar uma potência
que seja igual a um inteiro
dado.}}{Exercício: Encontrar uma potência que seja igual a um inteiro dado.}}\label{exercuxedcio-encontrar-uma-potuxeancia-que-seja-igual-a-um-inteiro-dado.}

Ler um inteiro \(n\) e encontrar dois inteiros \(b\) e
\(e, (0 < e < 6)\), tais que \(b^e = n\).

    \subsubsection{\texorpdfstring{Exercício: \emph{Gerar todos os números
primos menores que um valor
dado}}{Exercício: Gerar todos os números primos menores que um valor dado}}\label{exercuxedcio-gerar-todos-os-nuxfameros-primos-menores-que-um-valor-dado}

Ler um inteiro \(n\) e criar uma lista com todos os números primos
menores do que \(n\).

    \subsubsection{\texorpdfstring{Exercício: \emph{Ler uma sequência de
números e verificar se eles estão em ordem
crescente}}{Exercício: Ler uma sequência de números e verificar se eles estão em ordem crescente}}\label{exercuxedcio-ler-uma-sequuxeancia-de-nuxfameros-e-verificar-se-eles-estuxe3o-em-ordem-crescente}

Ler uma sequência de números e \textbf{\emph{depois}} verificar se eles
estão em ordem crescente.

    \subsubsection{\texorpdfstring{Exercício: \emph{Dada uma sequência de
números quaisquer, encontrar o par mais
próximo}}{Exercício: Dada uma sequência de números quaisquer, encontrar o par mais próximo}}\label{exercuxedcio-dada-uma-sequuxeancia-de-nuxfameros-quaisquer-encontrar-o-par-mais-pruxf3ximo}

Ler uma sequência de números \(S\) e \textbf{\emph{depois}} encontrar
\(a \in S\) e \(b \in S\), \(a \ne b\), tais que a distância entre eles,
isto é, \(\lvert a - b \rvert\), seja mínima.

    \subsubsection{\texorpdfstring{Exercício: \emph{Imprimir um triângulo
numérico}}{Exercício: Imprimir um triângulo numérico}}\label{exercuxedcio-imprimir-um-triuxe2ngulo-numuxe9rico}

Dado um inteiro positivo \(n\), imprimir um triângulo com \(n\) linhas e
o seguinte formato:
\texttt{1\ \ 1\ \ 2\ \ 1\ \ 2\ \ 3\ \ 1\ \ 2\ \ 3\ \ 4\ \ ...}

    \subsubsection{\texorpdfstring{Exercício: \emph{Imprimir um tapete
quadrado}}{Exercício: Imprimir um tapete quadrado}}\label{exercuxedcio-imprimir-um-tapete-quadrado}

Dado um inteiro positivo \(n\), imprimir um quadrado com \(n\) linhas e
colunas e o seguinte formato (supondo \(n = 5\)):

\begin{verbatim}
+ * * * *
* + * * *
* * + * *
* * * + *
* * * * +
\end{verbatim}


    % Add a bibliography block to the postdoc
    
    
    
    \end{document}

