
% Default to the notebook output style

    


% Inherit from the specified cell style.




    
\documentclass[11pt]{article}

    
    
    \usepackage[T1]{fontenc}
    % Nicer default font (+ math font) than Computer Modern for most use cases
    \usepackage{mathpazo}

    % Basic figure setup, for now with no caption control since it's done
    % automatically by Pandoc (which extracts ![](path) syntax from Markdown).
    \usepackage{graphicx}
    % We will generate all images so they have a width \maxwidth. This means
    % that they will get their normal width if they fit onto the page, but
    % are scaled down if they would overflow the margins.
    \makeatletter
    \def\maxwidth{\ifdim\Gin@nat@width>\linewidth\linewidth
    \else\Gin@nat@width\fi}
    \makeatother
    \let\Oldincludegraphics\includegraphics
    % Set max figure width to be 80% of text width, for now hardcoded.
    \renewcommand{\includegraphics}[1]{\Oldincludegraphics[width=.8\maxwidth]{#1}}
    % Ensure that by default, figures have no caption (until we provide a
    % proper Figure object with a Caption API and a way to capture that
    % in the conversion process - todo).
    \usepackage{caption}
    \DeclareCaptionLabelFormat{nolabel}{}
    \captionsetup{labelformat=nolabel}

    \usepackage{adjustbox} % Used to constrain images to a maximum size 
    \usepackage{xcolor} % Allow colors to be defined
    \usepackage{enumerate} % Needed for markdown enumerations to work
    \usepackage{geometry} % Used to adjust the document margins
    \usepackage{amsmath} % Equations
    \usepackage{amssymb} % Equations
    \usepackage{textcomp} % defines textquotesingle
    % Hack from http://tex.stackexchange.com/a/47451/13684:
    \AtBeginDocument{%
        \def\PYZsq{\textquotesingle}% Upright quotes in Pygmentized code
    }
    \usepackage{upquote} % Upright quotes for verbatim code
    \usepackage{eurosym} % defines \euro
    \usepackage[mathletters]{ucs} % Extended unicode (utf-8) support
    \usepackage[utf8x]{inputenc} % Allow utf-8 characters in the tex document
    \usepackage{fancyvrb} % verbatim replacement that allows latex
    \usepackage{grffile} % extends the file name processing of package graphics 
                         % to support a larger range 
    % The hyperref package gives us a pdf with properly built
    % internal navigation ('pdf bookmarks' for the table of contents,
    % internal cross-reference links, web links for URLs, etc.)
    \usepackage{hyperref}
    \usepackage{longtable} % longtable support required by pandoc >1.10
    \usepackage{booktabs}  % table support for pandoc > 1.12.2
    \usepackage[inline]{enumitem} % IRkernel/repr support (it uses the enumerate* environment)
    \usepackage[normalem]{ulem} % ulem is needed to support strikethroughs (\sout)
                                % normalem makes italics be italics, not underlines
    

    
    
    % Colors for the hyperref package
    \definecolor{urlcolor}{rgb}{0,.145,.698}
    \definecolor{linkcolor}{rgb}{.71,0.21,0.01}
    \definecolor{citecolor}{rgb}{.12,.54,.11}

    % ANSI colors
    \definecolor{ansi-black}{HTML}{3E424D}
    \definecolor{ansi-black-intense}{HTML}{282C36}
    \definecolor{ansi-red}{HTML}{E75C58}
    \definecolor{ansi-red-intense}{HTML}{B22B31}
    \definecolor{ansi-green}{HTML}{00A250}
    \definecolor{ansi-green-intense}{HTML}{007427}
    \definecolor{ansi-yellow}{HTML}{DDB62B}
    \definecolor{ansi-yellow-intense}{HTML}{B27D12}
    \definecolor{ansi-blue}{HTML}{208FFB}
    \definecolor{ansi-blue-intense}{HTML}{0065CA}
    \definecolor{ansi-magenta}{HTML}{D160C4}
    \definecolor{ansi-magenta-intense}{HTML}{A03196}
    \definecolor{ansi-cyan}{HTML}{60C6C8}
    \definecolor{ansi-cyan-intense}{HTML}{258F8F}
    \definecolor{ansi-white}{HTML}{C5C1B4}
    \definecolor{ansi-white-intense}{HTML}{A1A6B2}

    % commands and environments needed by pandoc snippets
    % extracted from the output of `pandoc -s`
    \providecommand{\tightlist}{%
      \setlength{\itemsep}{0pt}\setlength{\parskip}{0pt}}
    \DefineVerbatimEnvironment{Highlighting}{Verbatim}{commandchars=\\\{\}}
    % Add ',fontsize=\small' for more characters per line
    \newenvironment{Shaded}{}{}
    \newcommand{\KeywordTok}[1]{\textcolor[rgb]{0.00,0.44,0.13}{\textbf{{#1}}}}
    \newcommand{\DataTypeTok}[1]{\textcolor[rgb]{0.56,0.13,0.00}{{#1}}}
    \newcommand{\DecValTok}[1]{\textcolor[rgb]{0.25,0.63,0.44}{{#1}}}
    \newcommand{\BaseNTok}[1]{\textcolor[rgb]{0.25,0.63,0.44}{{#1}}}
    \newcommand{\FloatTok}[1]{\textcolor[rgb]{0.25,0.63,0.44}{{#1}}}
    \newcommand{\CharTok}[1]{\textcolor[rgb]{0.25,0.44,0.63}{{#1}}}
    \newcommand{\StringTok}[1]{\textcolor[rgb]{0.25,0.44,0.63}{{#1}}}
    \newcommand{\CommentTok}[1]{\textcolor[rgb]{0.38,0.63,0.69}{\textit{{#1}}}}
    \newcommand{\OtherTok}[1]{\textcolor[rgb]{0.00,0.44,0.13}{{#1}}}
    \newcommand{\AlertTok}[1]{\textcolor[rgb]{1.00,0.00,0.00}{\textbf{{#1}}}}
    \newcommand{\FunctionTok}[1]{\textcolor[rgb]{0.02,0.16,0.49}{{#1}}}
    \newcommand{\RegionMarkerTok}[1]{{#1}}
    \newcommand{\ErrorTok}[1]{\textcolor[rgb]{1.00,0.00,0.00}{\textbf{{#1}}}}
    \newcommand{\NormalTok}[1]{{#1}}
    
    % Additional commands for more recent versions of Pandoc
    \newcommand{\ConstantTok}[1]{\textcolor[rgb]{0.53,0.00,0.00}{{#1}}}
    \newcommand{\SpecialCharTok}[1]{\textcolor[rgb]{0.25,0.44,0.63}{{#1}}}
    \newcommand{\VerbatimStringTok}[1]{\textcolor[rgb]{0.25,0.44,0.63}{{#1}}}
    \newcommand{\SpecialStringTok}[1]{\textcolor[rgb]{0.73,0.40,0.53}{{#1}}}
    \newcommand{\ImportTok}[1]{{#1}}
    \newcommand{\DocumentationTok}[1]{\textcolor[rgb]{0.73,0.13,0.13}{\textit{{#1}}}}
    \newcommand{\AnnotationTok}[1]{\textcolor[rgb]{0.38,0.63,0.69}{\textbf{\textit{{#1}}}}}
    \newcommand{\CommentVarTok}[1]{\textcolor[rgb]{0.38,0.63,0.69}{\textbf{\textit{{#1}}}}}
    \newcommand{\VariableTok}[1]{\textcolor[rgb]{0.10,0.09,0.49}{{#1}}}
    \newcommand{\ControlFlowTok}[1]{\textcolor[rgb]{0.00,0.44,0.13}{\textbf{{#1}}}}
    \newcommand{\OperatorTok}[1]{\textcolor[rgb]{0.40,0.40,0.40}{{#1}}}
    \newcommand{\BuiltInTok}[1]{{#1}}
    \newcommand{\ExtensionTok}[1]{{#1}}
    \newcommand{\PreprocessorTok}[1]{\textcolor[rgb]{0.74,0.48,0.00}{{#1}}}
    \newcommand{\AttributeTok}[1]{\textcolor[rgb]{0.49,0.56,0.16}{{#1}}}
    \newcommand{\InformationTok}[1]{\textcolor[rgb]{0.38,0.63,0.69}{\textbf{\textit{{#1}}}}}
    \newcommand{\WarningTok}[1]{\textcolor[rgb]{0.38,0.63,0.69}{\textbf{\textit{{#1}}}}}
    
    
    % Define a nice break command that doesn't care if a line doesn't already
    % exist.
    \def\br{\hspace*{\fill} \\* }
    % Math Jax compatability definitions
    \def\gt{>}
    \def\lt{<}
    % Document parameters
    \title{Dicionários \\
           \small{MC102-2018s1-Aula13b-180412}
          }
    \author{Arthur J. Catto, PhD \\
            \small{ajcatto@g.unicamp.br}
            }
    \date{12 de abril de 2018}

    
    
    

    % Pygments definitions
    
\makeatletter
\def\PY@reset{\let\PY@it=\relax \let\PY@bf=\relax%
    \let\PY@ul=\relax \let\PY@tc=\relax%
    \let\PY@bc=\relax \let\PY@ff=\relax}
\def\PY@tok#1{\csname PY@tok@#1\endcsname}
\def\PY@toks#1+{\ifx\relax#1\empty\else%
    \PY@tok{#1}\expandafter\PY@toks\fi}
\def\PY@do#1{\PY@bc{\PY@tc{\PY@ul{%
    \PY@it{\PY@bf{\PY@ff{#1}}}}}}}
\def\PY#1#2{\PY@reset\PY@toks#1+\relax+\PY@do{#2}}

\expandafter\def\csname PY@tok@w\endcsname{\def\PY@tc##1{\textcolor[rgb]{0.73,0.73,0.73}{##1}}}
\expandafter\def\csname PY@tok@c\endcsname{\let\PY@it=\textit\def\PY@tc##1{\textcolor[rgb]{0.25,0.50,0.50}{##1}}}
\expandafter\def\csname PY@tok@cp\endcsname{\def\PY@tc##1{\textcolor[rgb]{0.74,0.48,0.00}{##1}}}
\expandafter\def\csname PY@tok@k\endcsname{\let\PY@bf=\textbf\def\PY@tc##1{\textcolor[rgb]{0.00,0.50,0.00}{##1}}}
\expandafter\def\csname PY@tok@kp\endcsname{\def\PY@tc##1{\textcolor[rgb]{0.00,0.50,0.00}{##1}}}
\expandafter\def\csname PY@tok@kt\endcsname{\def\PY@tc##1{\textcolor[rgb]{0.69,0.00,0.25}{##1}}}
\expandafter\def\csname PY@tok@o\endcsname{\def\PY@tc##1{\textcolor[rgb]{0.40,0.40,0.40}{##1}}}
\expandafter\def\csname PY@tok@ow\endcsname{\let\PY@bf=\textbf\def\PY@tc##1{\textcolor[rgb]{0.67,0.13,1.00}{##1}}}
\expandafter\def\csname PY@tok@nb\endcsname{\def\PY@tc##1{\textcolor[rgb]{0.00,0.50,0.00}{##1}}}
\expandafter\def\csname PY@tok@nf\endcsname{\def\PY@tc##1{\textcolor[rgb]{0.00,0.00,1.00}{##1}}}
\expandafter\def\csname PY@tok@nc\endcsname{\let\PY@bf=\textbf\def\PY@tc##1{\textcolor[rgb]{0.00,0.00,1.00}{##1}}}
\expandafter\def\csname PY@tok@nn\endcsname{\let\PY@bf=\textbf\def\PY@tc##1{\textcolor[rgb]{0.00,0.00,1.00}{##1}}}
\expandafter\def\csname PY@tok@ne\endcsname{\let\PY@bf=\textbf\def\PY@tc##1{\textcolor[rgb]{0.82,0.25,0.23}{##1}}}
\expandafter\def\csname PY@tok@nv\endcsname{\def\PY@tc##1{\textcolor[rgb]{0.10,0.09,0.49}{##1}}}
\expandafter\def\csname PY@tok@no\endcsname{\def\PY@tc##1{\textcolor[rgb]{0.53,0.00,0.00}{##1}}}
\expandafter\def\csname PY@tok@nl\endcsname{\def\PY@tc##1{\textcolor[rgb]{0.63,0.63,0.00}{##1}}}
\expandafter\def\csname PY@tok@ni\endcsname{\let\PY@bf=\textbf\def\PY@tc##1{\textcolor[rgb]{0.60,0.60,0.60}{##1}}}
\expandafter\def\csname PY@tok@na\endcsname{\def\PY@tc##1{\textcolor[rgb]{0.49,0.56,0.16}{##1}}}
\expandafter\def\csname PY@tok@nt\endcsname{\let\PY@bf=\textbf\def\PY@tc##1{\textcolor[rgb]{0.00,0.50,0.00}{##1}}}
\expandafter\def\csname PY@tok@nd\endcsname{\def\PY@tc##1{\textcolor[rgb]{0.67,0.13,1.00}{##1}}}
\expandafter\def\csname PY@tok@s\endcsname{\def\PY@tc##1{\textcolor[rgb]{0.73,0.13,0.13}{##1}}}
\expandafter\def\csname PY@tok@sd\endcsname{\let\PY@it=\textit\def\PY@tc##1{\textcolor[rgb]{0.73,0.13,0.13}{##1}}}
\expandafter\def\csname PY@tok@si\endcsname{\let\PY@bf=\textbf\def\PY@tc##1{\textcolor[rgb]{0.73,0.40,0.53}{##1}}}
\expandafter\def\csname PY@tok@se\endcsname{\let\PY@bf=\textbf\def\PY@tc##1{\textcolor[rgb]{0.73,0.40,0.13}{##1}}}
\expandafter\def\csname PY@tok@sr\endcsname{\def\PY@tc##1{\textcolor[rgb]{0.73,0.40,0.53}{##1}}}
\expandafter\def\csname PY@tok@ss\endcsname{\def\PY@tc##1{\textcolor[rgb]{0.10,0.09,0.49}{##1}}}
\expandafter\def\csname PY@tok@sx\endcsname{\def\PY@tc##1{\textcolor[rgb]{0.00,0.50,0.00}{##1}}}
\expandafter\def\csname PY@tok@m\endcsname{\def\PY@tc##1{\textcolor[rgb]{0.40,0.40,0.40}{##1}}}
\expandafter\def\csname PY@tok@gh\endcsname{\let\PY@bf=\textbf\def\PY@tc##1{\textcolor[rgb]{0.00,0.00,0.50}{##1}}}
\expandafter\def\csname PY@tok@gu\endcsname{\let\PY@bf=\textbf\def\PY@tc##1{\textcolor[rgb]{0.50,0.00,0.50}{##1}}}
\expandafter\def\csname PY@tok@gd\endcsname{\def\PY@tc##1{\textcolor[rgb]{0.63,0.00,0.00}{##1}}}
\expandafter\def\csname PY@tok@gi\endcsname{\def\PY@tc##1{\textcolor[rgb]{0.00,0.63,0.00}{##1}}}
\expandafter\def\csname PY@tok@gr\endcsname{\def\PY@tc##1{\textcolor[rgb]{1.00,0.00,0.00}{##1}}}
\expandafter\def\csname PY@tok@ge\endcsname{\let\PY@it=\textit}
\expandafter\def\csname PY@tok@gs\endcsname{\let\PY@bf=\textbf}
\expandafter\def\csname PY@tok@gp\endcsname{\let\PY@bf=\textbf\def\PY@tc##1{\textcolor[rgb]{0.00,0.00,0.50}{##1}}}
\expandafter\def\csname PY@tok@go\endcsname{\def\PY@tc##1{\textcolor[rgb]{0.53,0.53,0.53}{##1}}}
\expandafter\def\csname PY@tok@gt\endcsname{\def\PY@tc##1{\textcolor[rgb]{0.00,0.27,0.87}{##1}}}
\expandafter\def\csname PY@tok@err\endcsname{\def\PY@bc##1{\setlength{\fboxsep}{0pt}\fcolorbox[rgb]{1.00,0.00,0.00}{1,1,1}{\strut ##1}}}
\expandafter\def\csname PY@tok@kc\endcsname{\let\PY@bf=\textbf\def\PY@tc##1{\textcolor[rgb]{0.00,0.50,0.00}{##1}}}
\expandafter\def\csname PY@tok@kd\endcsname{\let\PY@bf=\textbf\def\PY@tc##1{\textcolor[rgb]{0.00,0.50,0.00}{##1}}}
\expandafter\def\csname PY@tok@kn\endcsname{\let\PY@bf=\textbf\def\PY@tc##1{\textcolor[rgb]{0.00,0.50,0.00}{##1}}}
\expandafter\def\csname PY@tok@kr\endcsname{\let\PY@bf=\textbf\def\PY@tc##1{\textcolor[rgb]{0.00,0.50,0.00}{##1}}}
\expandafter\def\csname PY@tok@bp\endcsname{\def\PY@tc##1{\textcolor[rgb]{0.00,0.50,0.00}{##1}}}
\expandafter\def\csname PY@tok@fm\endcsname{\def\PY@tc##1{\textcolor[rgb]{0.00,0.00,1.00}{##1}}}
\expandafter\def\csname PY@tok@vc\endcsname{\def\PY@tc##1{\textcolor[rgb]{0.10,0.09,0.49}{##1}}}
\expandafter\def\csname PY@tok@vg\endcsname{\def\PY@tc##1{\textcolor[rgb]{0.10,0.09,0.49}{##1}}}
\expandafter\def\csname PY@tok@vi\endcsname{\def\PY@tc##1{\textcolor[rgb]{0.10,0.09,0.49}{##1}}}
\expandafter\def\csname PY@tok@vm\endcsname{\def\PY@tc##1{\textcolor[rgb]{0.10,0.09,0.49}{##1}}}
\expandafter\def\csname PY@tok@sa\endcsname{\def\PY@tc##1{\textcolor[rgb]{0.73,0.13,0.13}{##1}}}
\expandafter\def\csname PY@tok@sb\endcsname{\def\PY@tc##1{\textcolor[rgb]{0.73,0.13,0.13}{##1}}}
\expandafter\def\csname PY@tok@sc\endcsname{\def\PY@tc##1{\textcolor[rgb]{0.73,0.13,0.13}{##1}}}
\expandafter\def\csname PY@tok@dl\endcsname{\def\PY@tc##1{\textcolor[rgb]{0.73,0.13,0.13}{##1}}}
\expandafter\def\csname PY@tok@s2\endcsname{\def\PY@tc##1{\textcolor[rgb]{0.73,0.13,0.13}{##1}}}
\expandafter\def\csname PY@tok@sh\endcsname{\def\PY@tc##1{\textcolor[rgb]{0.73,0.13,0.13}{##1}}}
\expandafter\def\csname PY@tok@s1\endcsname{\def\PY@tc##1{\textcolor[rgb]{0.73,0.13,0.13}{##1}}}
\expandafter\def\csname PY@tok@mb\endcsname{\def\PY@tc##1{\textcolor[rgb]{0.40,0.40,0.40}{##1}}}
\expandafter\def\csname PY@tok@mf\endcsname{\def\PY@tc##1{\textcolor[rgb]{0.40,0.40,0.40}{##1}}}
\expandafter\def\csname PY@tok@mh\endcsname{\def\PY@tc##1{\textcolor[rgb]{0.40,0.40,0.40}{##1}}}
\expandafter\def\csname PY@tok@mi\endcsname{\def\PY@tc##1{\textcolor[rgb]{0.40,0.40,0.40}{##1}}}
\expandafter\def\csname PY@tok@il\endcsname{\def\PY@tc##1{\textcolor[rgb]{0.40,0.40,0.40}{##1}}}
\expandafter\def\csname PY@tok@mo\endcsname{\def\PY@tc##1{\textcolor[rgb]{0.40,0.40,0.40}{##1}}}
\expandafter\def\csname PY@tok@ch\endcsname{\let\PY@it=\textit\def\PY@tc##1{\textcolor[rgb]{0.25,0.50,0.50}{##1}}}
\expandafter\def\csname PY@tok@cm\endcsname{\let\PY@it=\textit\def\PY@tc##1{\textcolor[rgb]{0.25,0.50,0.50}{##1}}}
\expandafter\def\csname PY@tok@cpf\endcsname{\let\PY@it=\textit\def\PY@tc##1{\textcolor[rgb]{0.25,0.50,0.50}{##1}}}
\expandafter\def\csname PY@tok@c1\endcsname{\let\PY@it=\textit\def\PY@tc##1{\textcolor[rgb]{0.25,0.50,0.50}{##1}}}
\expandafter\def\csname PY@tok@cs\endcsname{\let\PY@it=\textit\def\PY@tc##1{\textcolor[rgb]{0.25,0.50,0.50}{##1}}}

\def\PYZbs{\char`\\}
\def\PYZus{\char`\_}
\def\PYZob{\char`\{}
\def\PYZcb{\char`\}}
\def\PYZca{\char`\^}
\def\PYZam{\char`\&}
\def\PYZlt{\char`\<}
\def\PYZgt{\char`\>}
\def\PYZsh{\char`\#}
\def\PYZpc{\char`\%}
\def\PYZdl{\char`\$}
\def\PYZhy{\char`\-}
\def\PYZsq{\char`\'}
\def\PYZdq{\char`\"}
\def\PYZti{\char`\~}
% for compatibility with earlier versions
\def\PYZat{@}
\def\PYZlb{[}
\def\PYZrb{]}
\makeatother


    % Exact colors from NB
    \definecolor{incolor}{rgb}{0.0, 0.0, 0.5}
    \definecolor{outcolor}{rgb}{0.545, 0.0, 0.0}



    
    % Prevent overflowing lines due to hard-to-break entities
    \sloppy 
    % Setup hyperref package
    \hypersetup{
      breaklinks=true,  % so long urls are correctly broken across lines
      colorlinks=true,
      urlcolor=urlcolor,
      linkcolor=linkcolor,
      citecolor=citecolor,
      }
    % Slightly bigger margins than the latex defaults
    
    \geometry{verbose,tmargin=1in,bmargin=1in,lmargin=1in,rmargin=1in}
    
    

    \begin{document}
    
    
    \maketitle
    
    

    
%    \begin{Verbatim}[commandchars=\\\{\}]
%{\color{incolor}In [{\color{incolor}1}]:} \PY{k+kn}{from} \PY{n+nn}{IPython}\PY{n+nn}{.}\PY{n+nn}{core}\PY{n+nn}{.}\PY{n+nn}{interactiveshell} \PY{k}{import} \PY{n}{InteractiveShell}
%        \PY{n}{InteractiveShell}\PY{o}{.}\PY{n}{ast\PYZus{}node\PYZus{}interactivity} \PY{o}{=} \PY{l+s+s2}{\PYZdq{}}\PY{l+s+s2}{all}\PY{l+s+s2}{\PYZdq{}}
%\end{Verbatim}


    \section{Dicionários}\label{dicionuxe1rios}

    \subsection{O modelo}\label{o-modelo}

Listas em Python são um instrumento muito poderoso quando desejamos
extrair itens com base na sua posição em uma sequência.

Por exemplo, suponha a tabela com os nomes das cidades brasileiras mais
populosas publicada pelo IBGE:

    

    Se criarmos uma lista como...

    \begin{Verbatim}[commandchars=\\\{\}]
{\color{incolor}In [{\color{incolor}2}]:} \PY{n}{cidades} \PY{o}{=} \PY{p}{[}\PY{l+s+s1}{\PYZsq{}}\PY{l+s+s1}{São Paulo}\PY{l+s+s1}{\PYZsq{}}\PY{p}{,} \PY{l+s+s1}{\PYZsq{}}\PY{l+s+s1}{Rio de Janeiro}\PY{l+s+s1}{\PYZsq{}}\PY{p}{,} \PY{l+s+s1}{\PYZsq{}}\PY{l+s+s1}{Brasíia}\PY{l+s+s1}{\PYZsq{}}\PY{p}{,} \PY{l+s+s1}{\PYZsq{}}\PY{l+s+s1}{Salvador}\PY{l+s+s1}{\PYZsq{}}\PY{p}{,} \PY{l+s+s1}{\PYZsq{}}\PY{l+s+s1}{Fortaleza}\PY{l+s+s1}{\PYZsq{}}\PY{p}{]}
\end{Verbatim}


    ... é possível responder diretamente a uma pergunta como ``Qual a
\(n\)-ésima maior cidade do Brasil?''.

    \begin{Verbatim}[commandchars=\\\{\}]
{\color{incolor}In [{\color{incolor}11}]:} \PY{n}{n} \PY{o}{=} \PY{n+nb}{int}\PY{p}{(}\PY{n+nb}{input}\PY{p}{(}\PY{l+s+s2}{\PYZdq{}}\PY{l+s+s2}{Qual a posição da cidade desejada? }\PY{l+s+s2}{\PYZdq{}}\PY{p}{)}\PY{p}{)}
         \PY{n+nb}{print}\PY{p}{(}\PY{n}{f}\PY{l+s+s1}{\PYZsq{}}\PY{l+s+s1}{A }\PY{l+s+si}{\PYZob{}n\PYZcb{}}\PY{l+s+s1}{a maior cidade do Brasil é }\PY{l+s+si}{\PYZob{}cidades[n \PYZhy{} 1]\PYZcb{}}\PY{l+s+s1}{.}\PY{l+s+s1}{\PYZsq{}}\PY{p}{)}
\end{Verbatim}


    \begin{Verbatim}[commandchars=\\\{\}]
A 2a maior cidade do Brasil é Rio de Janeiro.

    \end{Verbatim}

    Numa lista, o valor do item que se acha numa determinada posição é
recuperado de forma direta. Essa operação tem ``custo'' constante, isto
é, independente da posição do item e do tamanho da lista.

    Suponha que quiséssemos também responder a ``Qual a população de
\(x\)?'', onde \(x\) é o nome de uma cidade na tabela.\\
Como fazer isso?

    Seria possível, por exemplo, criar uma segunda lista, com a população de
cada cidade, na mesma ordem.

    \begin{Verbatim}[commandchars=\\\{\}]
{\color{incolor}In [{\color{incolor}5}]:} \PY{n}{população} \PY{o}{=} \PY{p}{[}\PY{l+m+mi}{12106920}\PY{p}{,} \PY{l+m+mi}{6520266}\PY{p}{,} \PY{l+m+mi}{3039444}\PY{p}{,} \PY{l+m+mi}{2953986}\PY{p}{,} \PY{l+m+mi}{2627482}\PY{p}{]}
\end{Verbatim}


    Nesse caso, para obter a população de \(x\), será necessário fazer uma
busca, explicitamente ou usando uma função do sistema. Supondo que \(x\)
esteja na lista...

    \begin{Verbatim}[commandchars=\\\{\}]
{\color{incolor}In [{\color{incolor}12}]:} \PY{n}{cidade} \PY{o}{=} \PY{n+nb}{input}\PY{p}{(}\PY{l+s+s2}{\PYZdq{}}\PY{l+s+s2}{Qual a cidade desejada? }\PY{l+s+s2}{\PYZdq{}}\PY{p}{)}
         \PY{n}{i} \PY{o}{=} \PY{l+m+mi}{0}
         \PY{k}{while} \PY{n}{cidades}\PY{p}{[}\PY{n}{i}\PY{p}{]} \PY{o}{!=} \PY{n}{cidade}\PY{p}{:}
             \PY{n}{i} \PY{o}{+}\PY{o}{=} \PY{l+m+mi}{1}
         \PY{n+nb}{print}\PY{p}{(}\PY{n}{f}\PY{l+s+s1}{\PYZsq{}}\PY{l+s+s1}{A população de }\PY{l+s+si}{\PYZob{}cidade\PYZcb{}}\PY{l+s+s1}{ é }\PY{l+s+si}{\PYZob{}população[i]\PYZcb{}}\PY{l+s+s1}{.}\PY{l+s+s1}{\PYZsq{}}\PY{p}{)} 
\end{Verbatim}


    \begin{Verbatim}[commandchars=\\\{\}]
A população de Salvador é 2953986.

    \end{Verbatim}

    ou...

    \begin{Verbatim}[commandchars=\\\{\}]
{\color{incolor}In [{\color{incolor}13}]:} \PY{n}{cidade} \PY{o}{=} \PY{n+nb}{input}\PY{p}{(}\PY{l+s+s2}{\PYZdq{}}\PY{l+s+s2}{Qual a cidade desejada? }\PY{l+s+s2}{\PYZdq{}}\PY{p}{)}
         \PY{n+nb}{print}\PY{p}{(}\PY{n}{f}\PY{l+s+s1}{\PYZsq{}}\PY{l+s+s1}{A população de }\PY{l+s+si}{\PYZob{}cidade\PYZcb{}}\PY{l+s+s1}{ é }\PY{l+s+si}{\PYZob{}população[cidades.index(cidade)]\PYZcb{}}\PY{l+s+s1}{.}\PY{l+s+s1}{\PYZsq{}}\PY{p}{)}
\end{Verbatim}


    \begin{Verbatim}[commandchars=\\\{\}]
A população de Salvador é 2953986.

    \end{Verbatim}

    É possível também, por exemplo, criar uma lista de tuplas, reunindo a
cidade e a respectiva população.

    \begin{Verbatim}[commandchars=\\\{\}]
{\color{incolor}In [{\color{incolor}15}]:} \PY{n}{cid\PYZus{}pop} \PY{o}{=} \PY{p}{[}\PY{p}{(}\PY{l+s+s1}{\PYZsq{}}\PY{l+s+s1}{São Paulo}\PY{l+s+s1}{\PYZsq{}}\PY{p}{,} \PY{l+m+mi}{12106920}\PY{p}{)}\PY{p}{,} 
                    \PY{p}{(}\PY{l+s+s1}{\PYZsq{}}\PY{l+s+s1}{Rio de Janeiro}\PY{l+s+s1}{\PYZsq{}}\PY{p}{,} \PY{l+m+mi}{6520266}\PY{p}{)}\PY{p}{,} 
                    \PY{p}{(}\PY{l+s+s1}{\PYZsq{}}\PY{l+s+s1}{Brasíia}\PY{l+s+s1}{\PYZsq{}}\PY{p}{,} \PY{l+m+mi}{3039444}\PY{p}{)}\PY{p}{,} 
                    \PY{p}{(}\PY{l+s+s1}{\PYZsq{}}\PY{l+s+s1}{Salvador}\PY{l+s+s1}{\PYZsq{}}\PY{p}{,} \PY{l+m+mi}{2953986}\PY{p}{)}\PY{p}{,} 
                    \PY{p}{(}\PY{l+s+s1}{\PYZsq{}}\PY{l+s+s1}{Fortaleza}\PY{l+s+s1}{\PYZsq{}}\PY{p}{,} \PY{l+m+mi}{2627482}\PY{p}{)}
                   \PY{p}{]}
\end{Verbatim}


    E nesse caso, a resposta à pergunta pode ser obtida, por exemplo, por...

    \begin{Verbatim}[commandchars=\\\{\}]
{\color{incolor}In [{\color{incolor}17}]:} \PY{n}{x} \PY{o}{=} \PY{n+nb}{input}\PY{p}{(}\PY{l+s+s2}{\PYZdq{}}\PY{l+s+s2}{Qual a cidade desejada? }\PY{l+s+s2}{\PYZdq{}}\PY{p}{)}
         \PY{k}{for} \PY{n}{cidade}\PY{p}{,} \PY{n}{população} \PY{o+ow}{in} \PY{n}{cid\PYZus{}pop}\PY{p}{:}
             \PY{k}{if} \PY{n}{cidade} \PY{o}{==} \PY{n}{x}\PY{p}{:}
                 \PY{k}{break}
         \PY{n+nb}{print}\PY{p}{(}\PY{n}{f}\PY{l+s+s1}{\PYZsq{}}\PY{l+s+s1}{A população de }\PY{l+s+si}{\PYZob{}cidade\PYZcb{}}\PY{l+s+s1}{ é }\PY{l+s+si}{\PYZob{}população\PYZcb{}}\PY{l+s+s1}{.}\PY{l+s+s1}{\PYZsq{}}\PY{p}{)} 
\end{Verbatim}


    \begin{Verbatim}[commandchars=\\\{\}]
A população de Salvador é 2953986.

    \end{Verbatim}

    Em todos os casos, a operação de busca tem complexidade
\(\mathcal{O}(n)\), isto é o tempo de execução passa a ser proporcional
ao tamanho da lista.

    Um \textbf{\emph{dicionário}} em Python permite resolver esse problema
de uma forma muito mais eficiente.

Um \textbf{\emph{dicionário}} é uma \textbf{\emph{coleção não ordenada
de objetos}}, \textbf{\emph{mutável}}, \textbf{\emph{iterável}} e
\textbf{\emph{potencialmente heterogênea}}.

Os itens de um dicionários são acessíveis por \textbf{\emph{chaves}} e
não por índices, como nas estruturas sequenciais (listas, strings e
tuplas) que já estudamos.

    A implementação de dicionários usa uma técnica chmada \emph{hashing}
(que estudaremos mais à frente).

O uso de \emph{hashing} faz com que o tempo de busca de um determinado
valor seja potencialmente constante, em vez de crescer linearmente com o
tamanho da estrutura, como acontece nas listas.

    Um dicionário é representado por um conjunto de pares
\textbf{\emph{chave: valor}}, entre \textbf{\{} e \textbf{\}} e
separados por \textbf{\emph{vírgulas}}.

A \textbf{\emph{chave}} deve pertencer a um tipo imutável (isto é, nada
de \emph{listas} ou \emph{dicionários}), enquanto o
\textbf{\emph{valor}} pode ser de qualquer tipo.

    Por exemplo:

    \begin{Verbatim}[commandchars=\\\{\}]
{\color{incolor}In [{\color{incolor}24}]:} \PY{n}{cid2pop} \PY{o}{=} \PY{p}{\PYZob{}}\PY{l+s+s1}{\PYZsq{}}\PY{l+s+s1}{São Paulo}\PY{l+s+s1}{\PYZsq{}}\PY{p}{:} \PY{l+m+mi}{12106920}\PY{p}{,} 
                      \PY{l+s+s1}{\PYZsq{}}\PY{l+s+s1}{Rio de Janeiro}\PY{l+s+s1}{\PYZsq{}}\PY{p}{:} \PY{l+m+mi}{6520266}\PY{p}{,} 
                      \PY{l+s+s1}{\PYZsq{}}\PY{l+s+s1}{Brasília}\PY{l+s+s1}{\PYZsq{}}\PY{p}{:} \PY{l+m+mi}{3039400}\PY{p}{,}
                      \PY{l+s+s1}{\PYZsq{}}\PY{l+s+s1}{Salvador}\PY{l+s+s1}{\PYZsq{}}\PY{p}{:} \PY{l+m+mi}{2953986}\PY{p}{,} 
                      \PY{l+s+s1}{\PYZsq{}}\PY{l+s+s1}{Fortaleza}\PY{l+s+s1}{\PYZsq{}}\PY{p}{:} \PY{l+m+mi}{2627482}
                     \PY{p}{\PYZcb{}}
\end{Verbatim}


    E agora, para obter a população de \(x\), basta dizer

    \begin{Verbatim}[commandchars=\\\{\}]
{\color{incolor}In [{\color{incolor}25}]:} \PY{n}{x} \PY{o}{=} \PY{n+nb}{input}\PY{p}{(}\PY{l+s+s2}{\PYZdq{}}\PY{l+s+s2}{Qual a cidade desejada? }\PY{l+s+s2}{\PYZdq{}}\PY{p}{)}
         \PY{n+nb}{print}\PY{p}{(}\PY{n}{f}\PY{l+s+s1}{\PYZsq{}}\PY{l+s+s1}{A população de }\PY{l+s+si}{\PYZob{}cidade\PYZcb{}}\PY{l+s+s1}{ é }\PY{l+s+si}{\PYZob{}cid2pop[x]\PYZcb{}}\PY{l+s+s1}{.}\PY{l+s+s1}{\PYZsq{}}\PY{p}{)}
\end{Verbatim}


    \begin{Verbatim}[commandchars=\\\{\}]
A população de Salvador é 2953986.

    \end{Verbatim}

    Em compensação, os itens de um dicionário não são acessíveis por
índices...

    \begin{Verbatim}[commandchars=\\\{\}]
{\color{incolor}In [{\color{incolor}26}]:} \PY{n}{cid2pop}\PY{p}{[}\PY{l+m+mi}{1}\PY{p}{]}
\end{Verbatim}


    \begin{Verbatim}[commandchars=\\\{\}]

        ---------------------------------------------------------------------------

        KeyError                                  Traceback (most recent call last)

        <ipython-input-26-019bd9c87f36> in <module>()
    ----> 1 cid2pop[1]
    

        KeyError: 1

    \end{Verbatim}

    \subsection{Como criar um dicionário}\label{como-criar-um-dicionuxe1rio}

    O exemplo acima mostrou como criar um dicionário. Se quiséssemos um
dicionário \emph{dic} vazio faríamos

    \begin{Verbatim}[commandchars=\\\{\}]
{\color{incolor}In [{\color{incolor}53}]:} \PY{n}{dic} \PY{o}{=} \PY{p}{\PYZob{}}\PY{p}{\PYZcb{}}
         \PY{n}{dic}
\end{Verbatim}


\begin{Verbatim}[commandchars=\\\{\}]
{\color{outcolor}Out[{\color{outcolor}53}]:} \{\}
\end{Verbatim}
            
    \subsection{Como obter e modificar os itens de um
dicionário}\label{como-obter-e-modificar-os-itens-de-um-dicionuxe1rio}

Já vimos como obter o valor associado a uma chave.\\
A população de Brasília está errada. Vamos corrigi-la...

    \begin{Verbatim}[commandchars=\\\{\}]
{\color{incolor}In [{\color{incolor}27}]:} \PY{n}{cid2pop}\PY{p}{[}\PY{l+s+s1}{\PYZsq{}}\PY{l+s+s1}{Brasília}\PY{l+s+s1}{\PYZsq{}}\PY{p}{]} \PY{o}{=} \PY{l+m+mi}{3039444}
\end{Verbatim}


    O mesmo procedimento permite incluir novos itens no dicionário:

    \begin{Verbatim}[commandchars=\\\{\}]
{\color{incolor}In [{\color{incolor}28}]:} \PY{n}{cid2pop}\PY{p}{[}\PY{l+s+s1}{\PYZsq{}}\PY{l+s+s1}{Belo Horizonte}\PY{l+s+s1}{\PYZsq{}}\PY{p}{]} \PY{o}{=} \PY{l+m+mi}{2523000}
         \PY{n}{cid2pop}\PY{p}{[}\PY{l+s+s1}{\PYZsq{}}\PY{l+s+s1}{Manaus}\PY{l+s+s1}{\PYZsq{}}\PY{p}{]} \PY{o}{=} \PY{l+m+mi}{2130264}
\end{Verbatim}


    \begin{Verbatim}[commandchars=\\\{\}]
{\color{incolor}In [{\color{incolor}29}]:} \PY{n}{cid2pop}
\end{Verbatim}


\begin{Verbatim}[commandchars=\\\{\}]
{\color{outcolor}Out[{\color{outcolor}29}]:} \{'Belo Horizonte': 2523000,
          'Brasília': 3039444,
          'Fortaleza': 2627482,
          'Manaus': 2130264,
          'Rio de Janeiro': 6520266,
          'Salvador': 2953986,
          'São Paulo': 12106920\}
\end{Verbatim}
            
    É possível também incluir ou atualizar mais do que um novo item de uma
vez só, mesclando-se (\emph{merging}) um outro dicionário com o método
\textbf{update}.

    \begin{Verbatim}[commandchars=\\\{\}]
{\color{incolor}In [{\color{incolor}30}]:} \PY{n}{cid2pop}\PY{o}{.}\PY{n}{update}\PY{p}{(}\PY{p}{\PYZob{}}\PY{l+s+s1}{\PYZsq{}}\PY{l+s+s1}{Curitiba}\PY{l+s+s1}{\PYZsq{}}\PY{p}{:} \PY{l+m+mi}{1908359}\PY{p}{,} \PY{l+s+s1}{\PYZsq{}}\PY{l+s+s1}{Recife}\PY{l+s+s1}{\PYZsq{}}\PY{p}{:} \PY{l+m+mi}{1633697}\PY{p}{,} \PY{l+s+s1}{\PYZsq{}}\PY{l+s+s1}{Porto Alegre}\PY{l+s+s1}{\PYZsq{}}\PY{p}{:} \PY{l+m+mi}{1484941}\PY{p}{,} \PY{l+s+s1}{\PYZsq{}}\PY{l+s+s1}{Belo Horizonte}\PY{l+s+s1}{\PYZsq{}}\PY{p}{:} \PY{l+m+mi}{2523794}\PY{p}{\PYZcb{}}\PY{p}{)}
         \PY{n}{cid2pop}
\end{Verbatim}


\begin{Verbatim}[commandchars=\\\{\}]
{\color{outcolor}Out[{\color{outcolor}30}]:} \{'Belo Horizonte': 2523794,
          'Brasília': 3039444,
          'Curitiba': 1908359,
          'Fortaleza': 2627482,
          'Manaus': 2130264,
          'Porto Alegre': 1484941,
          'Recife': 1633697,
          'Rio de Janeiro': 6520266,
          'Salvador': 2953986,
          'São Paulo': 12106920\}
\end{Verbatim}
            
    O acesso com uma chave inexistente provoca um erro

    \begin{Verbatim}[commandchars=\\\{\}]
{\color{incolor}In [{\color{incolor}31}]:} \PY{n}{cid2pop}\PY{p}{[}\PY{l+s+s1}{\PYZsq{}}\PY{l+s+s1}{Campinas}\PY{l+s+s1}{\PYZsq{}}\PY{p}{]}
\end{Verbatim}


    \begin{Verbatim}[commandchars=\\\{\}]

        ---------------------------------------------------------------------------

        KeyError                                  Traceback (most recent call last)

        <ipython-input-31-e39e634706f3> in <module>()
    ----> 1 cid2pop['Campinas']
    

        KeyError: 'Campinas'

    \end{Verbatim}

    Para evitar o erro, usa-se o método \textbf{get} que aceita um parâmetro
com o valor que deve ser retornado, caso a chave consultada não seja
encontrada.

    \begin{Verbatim}[commandchars=\\\{\}]
{\color{incolor}In [{\color{incolor}32}]:} \PY{n}{cid2pop}\PY{o}{.}\PY{n}{get}\PY{p}{(}\PY{l+s+s1}{\PYZsq{}}\PY{l+s+s1}{Campinas}\PY{l+s+s1}{\PYZsq{}}\PY{p}{,} \PY{l+s+s1}{\PYZsq{}}\PY{l+s+s1}{Não sei...}\PY{l+s+s1}{\PYZsq{}}\PY{p}{)}
\end{Verbatim}


\begin{Verbatim}[commandchars=\\\{\}]
{\color{outcolor}Out[{\color{outcolor}32}]:} 'Não sei{\ldots}'
\end{Verbatim}
            
    \subsection{Como testar a existência de um item num
dicionário}\label{como-testar-a-existuxeancia-de-um-item-num-dicionuxe1rio}

O erro numa consulta com uma chave inexistente também pode ser evitado
protegendo-se a consulta por um teste de existência com os operadores
\textbf{in} e \textbf{not in}.

    \begin{Verbatim}[commandchars=\\\{\}]
{\color{incolor}In [{\color{incolor}33}]:} \PY{l+s+s1}{\PYZsq{}}\PY{l+s+s1}{Campinas}\PY{l+s+s1}{\PYZsq{}} \PY{o+ow}{in} \PY{n}{cid2pop}
\end{Verbatim}


\begin{Verbatim}[commandchars=\\\{\}]
{\color{outcolor}Out[{\color{outcolor}33}]:} False
\end{Verbatim}
            
    \begin{Verbatim}[commandchars=\\\{\}]
{\color{incolor}In [{\color{incolor}34}]:} \PY{l+s+s1}{\PYZsq{}}\PY{l+s+s1}{Campinas}\PY{l+s+s1}{\PYZsq{}} \PY{o+ow}{not} \PY{o+ow}{in} \PY{n}{cid2pop}
\end{Verbatim}


\begin{Verbatim}[commandchars=\\\{\}]
{\color{outcolor}Out[{\color{outcolor}34}]:} True
\end{Verbatim}
            
    ... o que nos permite fazer...

    \begin{Verbatim}[commandchars=\\\{\}]
{\color{incolor}In [{\color{incolor}35}]:} \PY{k}{if} \PY{l+s+s1}{\PYZsq{}}\PY{l+s+s1}{Campinas}\PY{l+s+s1}{\PYZsq{}} \PY{o+ow}{in} \PY{n}{cid2pop}\PY{p}{:}
             \PY{n}{resposta} \PY{o}{=} \PY{n}{cid2pop}\PY{p}{[}\PY{l+s+s1}{\PYZsq{}}\PY{l+s+s1}{Campinas}\PY{l+s+s1}{\PYZsq{}}\PY{p}{]}
         \PY{k}{else}\PY{p}{:}
             \PY{n}{resposta} \PY{o}{=} \PY{l+s+s1}{\PYZsq{}}\PY{l+s+s1}{não sei}\PY{l+s+s1}{\PYZsq{}}
         \PY{n}{resposta}
\end{Verbatim}


\begin{Verbatim}[commandchars=\\\{\}]
{\color{outcolor}Out[{\color{outcolor}35}]:} 'não sei'
\end{Verbatim}
            
    Esse comando também pode ser escrito de forma mais simples como

    \begin{Verbatim}[commandchars=\\\{\}]
{\color{incolor}In [{\color{incolor}36}]:} \PY{n}{resposta} \PY{o}{=} \PY{n}{cid2pop}\PY{p}{[}\PY{l+s+s1}{\PYZsq{}}\PY{l+s+s1}{Campinas}\PY{l+s+s1}{\PYZsq{}}\PY{p}{]} \PY{k}{if} \PY{l+s+s1}{\PYZsq{}}\PY{l+s+s1}{Campinas}\PY{l+s+s1}{\PYZsq{}} \PY{o+ow}{in} \PY{n}{cid2pop} \PY{k}{else} \PY{l+s+s1}{\PYZsq{}}\PY{l+s+s1}{não sei}\PY{l+s+s1}{\PYZsq{}}
         \PY{n}{resposta}
\end{Verbatim}


\begin{Verbatim}[commandchars=\\\{\}]
{\color{outcolor}Out[{\color{outcolor}36}]:} 'não sei'
\end{Verbatim}
            
    \subsection{Como remover itens de um
dicionário}\label{como-remover-itens-de-um-dicionuxe1rio}

Para remover um item de um dicionário usa-se o comando \textbf{del}

    \begin{Verbatim}[commandchars=\\\{\}]
{\color{incolor}In [{\color{incolor}37}]:} \PY{k}{del} \PY{n}{cid2pop}\PY{p}{[}\PY{l+s+s1}{\PYZsq{}}\PY{l+s+s1}{Manaus}\PY{l+s+s1}{\PYZsq{}}\PY{p}{]}
         \PY{n}{cid2pop}
\end{Verbatim}


\begin{Verbatim}[commandchars=\\\{\}]
{\color{outcolor}Out[{\color{outcolor}37}]:} \{'Belo Horizonte': 2523794,
          'Brasília': 3039444,
          'Curitiba': 1908359,
          'Fortaleza': 2627482,
          'Porto Alegre': 1484941,
          'Recife': 1633697,
          'Rio de Janeiro': 6520266,
          'Salvador': 2953986,
          'São Paulo': 12106920\}
\end{Verbatim}
            
    É possível obter o valor de um item e, ao mesmo tempo, removê-lo
usando-se o método \textbf{pop}.

    \begin{Verbatim}[commandchars=\\\{\}]
{\color{incolor}In [{\color{incolor}39}]:} \PY{n+nb}{print}\PY{p}{(}\PY{n}{cid2pop}\PY{o}{.}\PY{n}{pop}\PY{p}{(}\PY{l+s+s1}{\PYZsq{}}\PY{l+s+s1}{Fortaleza}\PY{l+s+s1}{\PYZsq{}}\PY{p}{)}\PY{p}{)}
\end{Verbatim}


    \begin{Verbatim}[commandchars=\\\{\}]
2627482

    \end{Verbatim}

    \begin{Verbatim}[commandchars=\\\{\}]
{\color{incolor}In [{\color{incolor}40}]:} \PY{n}{cid2pop}
\end{Verbatim}


\begin{Verbatim}[commandchars=\\\{\}]
{\color{outcolor}Out[{\color{outcolor}40}]:} \{'Belo Horizonte': 2523794,
          'Brasília': 3039444,
          'Curitiba': 1908359,
          'Porto Alegre': 1484941,
          'Recife': 1633697,
          'Rio de Janeiro': 6520266,
          'Salvador': 2953986,
          'São Paulo': 12106920\}
\end{Verbatim}
            
    A tentativa de remoção de uma chave inexistente gera um erro...

    \begin{Verbatim}[commandchars=\\\{\}]
{\color{incolor}In [{\color{incolor}41}]:} \PY{n+nb}{print}\PY{p}{(}\PY{n}{cid2pop}\PY{o}{.}\PY{n}{pop}\PY{p}{(}\PY{l+s+s1}{\PYZsq{}}\PY{l+s+s1}{Campinas}\PY{l+s+s1}{\PYZsq{}}\PY{p}{)}\PY{p}{)}
\end{Verbatim}


    \begin{Verbatim}[commandchars=\\\{\}]

        ---------------------------------------------------------------------------

        KeyError                                  Traceback (most recent call last)

        <ipython-input-41-bd09199dcbe3> in <module>()
    ----> 1 print(cid2pop.pop('Campinas'))
    

        KeyError: 'Campinas'

    \end{Verbatim}

    É possível evitar o erro na remoção de uma chave inexistente dando
\emph{None} como um segundo argumento para \textbf{pop}.

    \begin{Verbatim}[commandchars=\\\{\}]
{\color{incolor}In [{\color{incolor}42}]:} \PY{n+nb}{print}\PY{p}{(}\PY{n}{população}\PY{o}{.}\PY{n}{pop}\PY{p}{(}\PY{l+s+s1}{\PYZsq{}}\PY{l+s+s1}{Campinas}\PY{l+s+s1}{\PYZsq{}}\PY{p}{,} \PY{k+kc}{None}\PY{p}{)}\PY{p}{)}
\end{Verbatim}


    \begin{Verbatim}[commandchars=\\\{\}]
None

    \end{Verbatim}

    Para remover todas os itens de um dicionário usa-se o método
\textbf{clear}.

\textbf{Clear} esvazia o dicionário associado à variável à qual ele é
aplicado.

\begin{itemize}
\tightlist
\item
  A variável continua associada ao mesmo objeto (isto é, seu \textbf{id}
  não se altera).
\item
  Apenas o valor do objeto é que foi alterado (neste caso para
  \emph{vazio}).
\end{itemize}

    \begin{Verbatim}[commandchars=\\\{\}]
{\color{incolor}In [{\color{incolor}43}]:} \PY{n}{dic} \PY{o}{=} \PY{p}{\PYZob{}}\PY{l+s+s1}{\PYZsq{}}\PY{l+s+s1}{a}\PY{l+s+s1}{\PYZsq{}}\PY{p}{:} \PY{l+m+mi}{123}\PY{p}{,} \PY{l+s+s1}{\PYZsq{}}\PY{l+s+s1}{b}\PY{l+s+s1}{\PYZsq{}}\PY{p}{:} \PY{l+m+mi}{456}\PY{p}{,} \PY{l+s+s1}{\PYZsq{}}\PY{l+s+s1}{c}\PY{l+s+s1}{\PYZsq{}}\PY{p}{:} \PY{l+m+mi}{789}\PY{p}{\PYZcb{}}
         \PY{n+nb}{id}\PY{p}{(}\PY{n}{dic}\PY{p}{)}\PY{p}{,} \PY{n}{dic}
\end{Verbatim}


\begin{Verbatim}[commandchars=\\\{\}]
{\color{outcolor}Out[{\color{outcolor}43}]:} (4463474728, \{'a': 123, 'b': 456, 'c': 789\})
\end{Verbatim}
            
    \begin{Verbatim}[commandchars=\\\{\}]
{\color{incolor}In [{\color{incolor}44}]:} \PY{n}{dic}\PY{o}{.}\PY{n}{clear}\PY{p}{(}\PY{p}{)}
         \PY{n+nb}{id}\PY{p}{(}\PY{n}{dic}\PY{p}{)}\PY{p}{,} \PY{n}{dic}
\end{Verbatim}


\begin{Verbatim}[commandchars=\\\{\}]
{\color{outcolor}Out[{\color{outcolor}44}]:} (4463474728, \{\})
\end{Verbatim}
            
    Note que isto não é o mesmo que atribuir \textbf{\{\}} à variável. Neste
caso a variável deixa de estar associada ao antigo dicionário (que
continua existindo) e passa a estar associada a outro dicionário (neste
caso um dicionário vazio). Com isso, o \textbf{id} da variável muda,
como mostram os exemplos a seguir.

    \begin{Verbatim}[commandchars=\\\{\}]
{\color{incolor}In [{\color{incolor}45}]:} \PY{n}{dic} \PY{o}{=} \PY{p}{\PYZob{}}\PY{l+s+s1}{\PYZsq{}}\PY{l+s+s1}{a}\PY{l+s+s1}{\PYZsq{}}\PY{p}{:} \PY{l+m+mi}{123}\PY{p}{,} \PY{l+s+s1}{\PYZsq{}}\PY{l+s+s1}{b}\PY{l+s+s1}{\PYZsq{}}\PY{p}{:} \PY{l+m+mi}{456}\PY{p}{,} \PY{l+s+s1}{\PYZsq{}}\PY{l+s+s1}{c}\PY{l+s+s1}{\PYZsq{}}\PY{p}{:} \PY{l+m+mi}{789}\PY{p}{\PYZcb{}}
         \PY{n+nb}{id}\PY{p}{(}\PY{n}{dic}\PY{p}{)}\PY{p}{,} \PY{n}{dic}
\end{Verbatim}


\begin{Verbatim}[commandchars=\\\{\}]
{\color{outcolor}Out[{\color{outcolor}45}]:} (4435158072, \{'a': 123, 'b': 456, 'c': 789\})
\end{Verbatim}
            
    \begin{Verbatim}[commandchars=\\\{\}]
{\color{incolor}In [{\color{incolor}46}]:} \PY{n}{dic} \PY{o}{=} \PY{p}{\PYZob{}}\PY{p}{\PYZcb{}}
         \PY{n+nb}{id}\PY{p}{(}\PY{n}{dic}\PY{p}{)}\PY{p}{,} \PY{n}{dic}
\end{Verbatim}


\begin{Verbatim}[commandchars=\\\{\}]
{\color{outcolor}Out[{\color{outcolor}46}]:} (4435383136, \{\})
\end{Verbatim}
            
    \subsection{Iterações sobre
dicionários}\label{iterauxe7uxf5es-sobre-dicionuxe1rios}

É possível iterar facilmente sobre as chaves ou valores de um dicionário

    \begin{Verbatim}[commandchars=\\\{\}]
{\color{incolor}In [{\color{incolor}63}]:} \PY{n}{cid2pop} \PY{o}{=} \PY{p}{\PYZob{}}\PY{l+s+s1}{\PYZsq{}}\PY{l+s+s1}{Salvador}\PY{l+s+s1}{\PYZsq{}}\PY{p}{:} \PY{l+m+mi}{2954000}\PY{p}{,}
                      \PY{l+s+s1}{\PYZsq{}}\PY{l+s+s1}{Fortaleza}\PY{l+s+s1}{\PYZsq{}}\PY{p}{:} \PY{l+m+mi}{2627482}\PY{p}{,}
                      \PY{l+s+s1}{\PYZsq{}}\PY{l+s+s1}{Belo Horizonte}\PY{l+s+s1}{\PYZsq{}}\PY{p}{:} \PY{l+m+mi}{2523794}\PY{p}{,}
                      \PY{l+s+s1}{\PYZsq{}}\PY{l+s+s1}{São Paulo}\PY{l+s+s1}{\PYZsq{}}\PY{p}{:} \PY{l+m+mi}{12106920}\PY{p}{,}
                      \PY{l+s+s1}{\PYZsq{}}\PY{l+s+s1}{Brasília}\PY{l+s+s1}{\PYZsq{}}\PY{p}{:} \PY{l+m+mi}{3039444}\PY{p}{,}
                      \PY{l+s+s1}{\PYZsq{}}\PY{l+s+s1}{Rio de Janeiro}\PY{l+s+s1}{\PYZsq{}}\PY{p}{:} \PY{l+m+mi}{6520266}\PY{p}{,}
                      \PY{l+s+s1}{\PYZsq{}}\PY{l+s+s1}{Manaus}\PY{l+s+s1}{\PYZsq{}}\PY{p}{:} \PY{l+m+mi}{2130264}\PY{p}{\PYZcb{}}
\end{Verbatim}


    \begin{Verbatim}[commandchars=\\\{\}]
{\color{incolor}In [{\color{incolor}51}]:} \PY{k}{for} \PY{n}{x} \PY{o+ow}{in} \PY{n}{cid2pop}\PY{p}{:}
             \PY{n+nb}{print}\PY{p}{(}\PY{n}{f}\PY{l+s+s1}{\PYZsq{}}\PY{l+s+si}{\PYZob{}x:16\PYZcb{}}\PY{l+s+s1}{  }\PY{l+s+si}{\PYZob{}cid2pop[x]:8\PYZcb{}}\PY{l+s+s1}{\PYZsq{}}\PY{p}{)}
\end{Verbatim}


    \begin{Verbatim}[commandchars=\\\{\}]
São Paulo         12106920
Rio de Janeiro     6520266
Brasília           3039444
Salvador           2953986
Belo Horizonte     2523794
Curitiba           1908359
Recife             1633697
Porto Alegre       1484941

    \end{Verbatim}

    O mesmo resultado pode ser conseguido de um jeito mais \emph{pythoniano}
por

    \begin{Verbatim}[commandchars=\\\{\}]
{\color{incolor}In [{\color{incolor}50}]:} \PY{k}{for} \PY{n}{x}\PY{p}{,} \PY{n}{popx} \PY{o+ow}{in} \PY{n}{cid2pop}\PY{o}{.}\PY{n}{items}\PY{p}{(}\PY{p}{)}\PY{p}{:}
             \PY{n+nb}{print}\PY{p}{(}\PY{n}{f}\PY{l+s+s1}{\PYZsq{}}\PY{l+s+si}{\PYZob{}x:16\PYZcb{}}\PY{l+s+s1}{  }\PY{l+s+si}{\PYZob{}popx:8\PYZcb{}}\PY{l+s+s1}{\PYZsq{}}\PY{p}{)}
\end{Verbatim}


    \begin{Verbatim}[commandchars=\\\{\}]
São Paulo         12106920
Rio de Janeiro     6520266
Brasília           3039444
Salvador           2953986
Belo Horizonte     2523794
Curitiba           1908359
Recife             1633697
Porto Alegre       1484941

    \end{Verbatim}

    \subsection{Exemplos}\label{exemplos}

    \subsubsection{Quais as três cidades brasileiras mais
populosas?}\label{quais-as-truxeas-cidades-brasileiras-mais-populosas}

    \begin{Verbatim}[commandchars=\\\{\}]
{\color{incolor}In [{\color{incolor}55}]:} \PY{n}{cid2pop} \PY{o}{=} \PY{p}{\PYZob{}}\PY{l+s+s1}{\PYZsq{}}\PY{l+s+s1}{Salvador}\PY{l+s+s1}{\PYZsq{}}\PY{p}{:} \PY{l+m+mi}{2954000}\PY{p}{,}
                    \PY{l+s+s1}{\PYZsq{}}\PY{l+s+s1}{Fortaleza}\PY{l+s+s1}{\PYZsq{}}\PY{p}{:} \PY{l+m+mi}{2627482}\PY{p}{,}
                    \PY{l+s+s1}{\PYZsq{}}\PY{l+s+s1}{Belo Horizonte}\PY{l+s+s1}{\PYZsq{}}\PY{p}{:} \PY{l+m+mi}{2523794}\PY{p}{,}
                    \PY{l+s+s1}{\PYZsq{}}\PY{l+s+s1}{São Paulo}\PY{l+s+s1}{\PYZsq{}}\PY{p}{:} \PY{l+m+mi}{12106920}\PY{p}{,}
                    \PY{l+s+s1}{\PYZsq{}}\PY{l+s+s1}{Brasília}\PY{l+s+s1}{\PYZsq{}}\PY{p}{:} \PY{l+m+mi}{3039444}\PY{p}{,}
                    \PY{l+s+s1}{\PYZsq{}}\PY{l+s+s1}{Rio de Janeiro}\PY{l+s+s1}{\PYZsq{}}\PY{p}{:} \PY{l+m+mi}{6520266}\PY{p}{,}
                    \PY{l+s+s1}{\PYZsq{}}\PY{l+s+s1}{Manaus}\PY{l+s+s1}{\PYZsq{}}\PY{p}{:} \PY{l+m+mi}{2130264}\PY{p}{,}
                    \PY{l+s+s1}{\PYZsq{}}\PY{l+s+s1}{Curitiba}\PY{l+s+s1}{\PYZsq{}}\PY{p}{:} \PY{l+m+mi}{1908359}\PY{p}{,}
                    \PY{l+s+s1}{\PYZsq{}}\PY{l+s+s1}{Recife}\PY{l+s+s1}{\PYZsq{}}\PY{p}{:} \PY{l+m+mi}{1633697}\PY{p}{,}
                    \PY{l+s+s1}{\PYZsq{}}\PY{l+s+s1}{Porto Alegre}\PY{l+s+s1}{\PYZsq{}}\PY{p}{:} \PY{l+m+mi}{1484941}
                   \PY{p}{\PYZcb{}}
\end{Verbatim}


    \begin{Verbatim}[commandchars=\\\{\}]
{\color{incolor}In [{\color{incolor}56}]:} \PY{k+kn}{from} \PY{n+nn}{operator} \PY{k}{import} \PY{n}{itemgetter}
         
         \PY{n}{scid2pop} \PY{o}{=} \PY{n+nb}{sorted}\PY{p}{(}\PY{n}{cid2pop}\PY{o}{.}\PY{n}{items}\PY{p}{(}\PY{p}{)}\PY{p}{,} \PY{n}{key}\PY{o}{=}\PY{n}{itemgetter}\PY{p}{(}\PY{l+m+mi}{1}\PY{p}{)}\PY{p}{,} \PY{n}{reverse}\PY{o}{=}\PY{k+kc}{True}\PY{p}{)}
         \PY{n}{scid2pop}\PY{p}{[}\PY{p}{:}\PY{l+m+mi}{3}\PY{p}{]}
\end{Verbatim}


\begin{Verbatim}[commandchars=\\\{\}]
{\color{outcolor}Out[{\color{outcolor}56}]:} [('São Paulo', 12106920), ('Rio de Janeiro', 6520266), ('Brasília', 3039444)]
\end{Verbatim}
            
    \subsubsection{Quais cidades brasileiras têm mais do que 2 milhões de
habitantes?}\label{quais-cidades-brasileiras-tuxeam-mais-do-que-2-milhuxf5es-de-habitantes}

    \begin{Verbatim}[commandchars=\\\{\}]
{\color{incolor}In [{\color{incolor}57}]:} \PY{n}{cid2pop} \PY{o}{=} \PY{p}{\PYZob{}}\PY{l+s+s1}{\PYZsq{}}\PY{l+s+s1}{Salvador}\PY{l+s+s1}{\PYZsq{}}\PY{p}{:} \PY{l+m+mi}{2954000}\PY{p}{,}
                    \PY{l+s+s1}{\PYZsq{}}\PY{l+s+s1}{Fortaleza}\PY{l+s+s1}{\PYZsq{}}\PY{p}{:} \PY{l+m+mi}{2627482}\PY{p}{,}
                    \PY{l+s+s1}{\PYZsq{}}\PY{l+s+s1}{Belo Horizonte}\PY{l+s+s1}{\PYZsq{}}\PY{p}{:} \PY{l+m+mi}{2523794}\PY{p}{,}
                    \PY{l+s+s1}{\PYZsq{}}\PY{l+s+s1}{São Paulo}\PY{l+s+s1}{\PYZsq{}}\PY{p}{:} \PY{l+m+mi}{12106920}\PY{p}{,}
                    \PY{l+s+s1}{\PYZsq{}}\PY{l+s+s1}{Brasília}\PY{l+s+s1}{\PYZsq{}}\PY{p}{:} \PY{l+m+mi}{3039444}\PY{p}{,}
                    \PY{l+s+s1}{\PYZsq{}}\PY{l+s+s1}{Rio de Janeiro}\PY{l+s+s1}{\PYZsq{}}\PY{p}{:} \PY{l+m+mi}{6520266}\PY{p}{,}
                    \PY{l+s+s1}{\PYZsq{}}\PY{l+s+s1}{Manaus}\PY{l+s+s1}{\PYZsq{}}\PY{p}{:} \PY{l+m+mi}{2130264}\PY{p}{,}
                    \PY{l+s+s1}{\PYZsq{}}\PY{l+s+s1}{Curitiba}\PY{l+s+s1}{\PYZsq{}}\PY{p}{:} \PY{l+m+mi}{1908359}\PY{p}{,}
                    \PY{l+s+s1}{\PYZsq{}}\PY{l+s+s1}{Recife}\PY{l+s+s1}{\PYZsq{}}\PY{p}{:} \PY{l+m+mi}{1633697}\PY{p}{,}
                    \PY{l+s+s1}{\PYZsq{}}\PY{l+s+s1}{Porto Alegre}\PY{l+s+s1}{\PYZsq{}}\PY{p}{:} \PY{l+m+mi}{1484941}
                   \PY{p}{\PYZcb{}}
\end{Verbatim}


    \begin{Verbatim}[commandchars=\\\{\}]
{\color{incolor}In [{\color{incolor}60}]:} \PY{k+kn}{from} \PY{n+nn}{operator} \PY{k}{import} \PY{n}{itemgetter}
         
         \PY{n}{grandes} \PY{o}{=} \PY{p}{[}\PY{p}{(}\PY{n}{rm}\PY{p}{,} \PY{n}{pop}\PY{p}{)} \PY{k}{for} \PY{n}{rm}\PY{p}{,} \PY{n}{pop} \PY{o+ow}{in} \PY{n}{cid2pop}\PY{o}{.}\PY{n}{items}\PY{p}{(}\PY{p}{)} \PY{k}{if} \PY{n}{pop} \PY{o}{\PYZgt{}} \PY{l+m+mi}{2000000}\PY{p}{]}
         \PY{n}{grandes} \PY{o}{=} \PY{n+nb}{sorted}\PY{p}{(}\PY{n}{grandes}\PY{p}{,} \PY{n}{key}\PY{o}{=}\PY{n}{itemgetter}\PY{p}{(}\PY{l+m+mi}{1}\PY{p}{)}\PY{p}{,} \PY{n}{reverse}\PY{o}{=}\PY{k+kc}{True}\PY{p}{)}
         \PY{n}{grandes}
\end{Verbatim}


\begin{Verbatim}[commandchars=\\\{\}]
{\color{outcolor}Out[{\color{outcolor}60}]:} [('São Paulo', 12106920),
          ('Rio de Janeiro', 6520266),
          ('Brasília', 3039444),
          ('Salvador', 2954000),
          ('Fortaleza', 2627482),
          ('Belo Horizonte', 2523794),
          ('Manaus', 2130264)]
\end{Verbatim}
            
    \subsubsection{Dentre as cidades brasileiras com mais de 1,5 milhões de
habitantes, quais as duas com nome mais
curto?}\label{dentre-as-cidades-brasileiras-com-mais-de-15-milhuxf5es-de-habitantes-quais-as-duas-com-nome-mais-curto}

    \begin{Verbatim}[commandchars=\\\{\}]
{\color{incolor}In [{\color{incolor}61}]:} \PY{n}{cid2pop} \PY{o}{=} \PY{p}{\PYZob{}}\PY{l+s+s1}{\PYZsq{}}\PY{l+s+s1}{São Paulo}\PY{l+s+s1}{\PYZsq{}}\PY{p}{:} \PY{l+m+mi}{12106920}\PY{p}{,} \PY{l+s+s1}{\PYZsq{}}\PY{l+s+s1}{Rio de Janeiro}\PY{l+s+s1}{\PYZsq{}}\PY{p}{:} \PY{l+m+mi}{6520266}\PY{p}{,}
                    \PY{l+s+s1}{\PYZsq{}}\PY{l+s+s1}{Brasília}\PY{l+s+s1}{\PYZsq{}}\PY{p}{:} \PY{l+m+mi}{3039444}\PY{p}{,}   \PY{l+s+s1}{\PYZsq{}}\PY{l+s+s1}{Salvador}\PY{l+s+s1}{\PYZsq{}}\PY{p}{:} \PY{l+m+mi}{2953986}\PY{p}{,}
                    \PY{l+s+s1}{\PYZsq{}}\PY{l+s+s1}{Fortaleza}\PY{l+s+s1}{\PYZsq{}}\PY{p}{:} \PY{l+m+mi}{2627482}\PY{p}{,}  \PY{l+s+s1}{\PYZsq{}}\PY{l+s+s1}{Belo Horizonte}\PY{l+s+s1}{\PYZsq{}}\PY{p}{:} \PY{l+m+mi}{2523794}\PY{p}{,}
                    \PY{l+s+s1}{\PYZsq{}}\PY{l+s+s1}{Manaus}\PY{l+s+s1}{\PYZsq{}}\PY{p}{:} \PY{l+m+mi}{2130264}\PY{p}{,}     \PY{l+s+s1}{\PYZsq{}}\PY{l+s+s1}{Curitiba}\PY{l+s+s1}{\PYZsq{}}\PY{p}{:} \PY{l+m+mi}{1908359}\PY{p}{,}
                    \PY{l+s+s1}{\PYZsq{}}\PY{l+s+s1}{Recife}\PY{l+s+s1}{\PYZsq{}}\PY{p}{:} \PY{l+m+mi}{1633697}\PY{p}{,}     \PY{l+s+s1}{\PYZsq{}}\PY{l+s+s1}{Porto Alegre}\PY{l+s+s1}{\PYZsq{}}\PY{p}{:} \PY{l+m+mi}{1484941}\PY{p}{,}
                    \PY{l+s+s1}{\PYZsq{}}\PY{l+s+s1}{Goiânia}\PY{l+s+s1}{\PYZsq{}}\PY{p}{:} \PY{l+m+mi}{1466105}\PY{p}{,}    \PY{l+s+s1}{\PYZsq{}}\PY{l+s+s1}{Belém}\PY{l+s+s1}{\PYZsq{}}\PY{p}{:} \PY{l+m+mi}{1452275}\PY{p}{,}
                    \PY{l+s+s1}{\PYZsq{}}\PY{l+s+s1}{Guarulhos}\PY{l+s+s1}{\PYZsq{}}\PY{p}{:} \PY{l+m+mi}{1349113}\PY{p}{,}  \PY{l+s+s1}{\PYZsq{}}\PY{l+s+s1}{Campinas}\PY{l+s+s1}{\PYZsq{}}\PY{p}{:} \PY{l+m+mi}{1182429}\PY{p}{,}
                    \PY{l+s+s1}{\PYZsq{}}\PY{l+s+s1}{São Luís}\PY{l+s+s1}{\PYZsq{}}\PY{p}{:} \PY{l+m+mi}{1091868}\PY{p}{,}   \PY{l+s+s1}{\PYZsq{}}\PY{l+s+s1}{São Gonçalo}\PY{l+s+s1}{\PYZsq{}}\PY{p}{:} \PY{l+m+mi}{1049826}\PY{p}{,}
                    \PY{l+s+s1}{\PYZsq{}}\PY{l+s+s1}{Maceió}\PY{l+s+s1}{\PYZsq{}}\PY{p}{:} \PY{l+m+mi}{1029129}
                   \PY{p}{\PYZcb{}}
\end{Verbatim}


    \begin{Verbatim}[commandchars=\\\{\}]
{\color{incolor}In [{\color{incolor}62}]:} \PY{k+kn}{from} \PY{n+nn}{operator} \PY{k}{import} \PY{n}{itemgetter}
         
         \PY{n}{tam\PYZus{}nomes} \PY{o}{=} \PY{p}{[}\PY{p}{(}\PY{n}{nome}\PY{p}{,} \PY{n}{pop}\PY{p}{,} \PY{n+nb}{len}\PY{p}{(}\PY{n}{nome}\PY{p}{)}\PY{p}{)} \PY{k}{for} \PY{n}{nome}\PY{p}{,} \PY{n}{pop} \PY{o+ow}{in} \PY{n}{cid2pop}\PY{o}{.}\PY{n}{items}\PY{p}{(}\PY{p}{)} \PY{k}{if} \PY{n}{pop} \PY{o}{\PYZgt{}} \PY{l+m+mi}{1500000}\PY{p}{]}
         \PY{n}{tam\PYZus{}nomes} \PY{o}{=} \PY{n+nb}{sorted}\PY{p}{(}\PY{n}{tam\PYZus{}nomes}\PY{p}{,} \PY{n}{key}\PY{o}{=}\PY{n}{itemgetter}\PY{p}{(}\PY{l+m+mi}{2}\PY{p}{)}\PY{p}{)}
         \PY{n}{tam\PYZus{}nomes}
         \PY{n}{tam\PYZus{}nomes}\PY{p}{[}\PY{p}{:}\PY{l+m+mi}{2}\PY{p}{]}
\end{Verbatim}


\begin{Verbatim}[commandchars=\\\{\}]
{\color{outcolor}Out[{\color{outcolor}62}]:} [('Manaus', 2130264, 6),
          ('Recife', 1633697, 6),
          ('Brasília', 3039444, 8),
          ('Salvador', 2953986, 8),
          ('Curitiba', 1908359, 8),
          ('São Paulo', 12106920, 9),
          ('Fortaleza', 2627482, 9),
          ('Rio de Janeiro', 6520266, 14),
          ('Belo Horizonte', 2523794, 14)]
\end{Verbatim}
            
\begin{Verbatim}[commandchars=\\\{\}]
{\color{outcolor}Out[{\color{outcolor}62}]:} [('Manaus', 2130264, 6), ('Recife', 1633697, 6)]
\end{Verbatim}
            

    % Add a bibliography block to the postdoc
    
    
    
    \end{document}
